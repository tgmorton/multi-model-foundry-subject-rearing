% Simplified version for non-linguistic audience
\documentclass[aspectratio=169]{beamer}

% --- Theme ---
\makeatletter
\def\input@path{{beamertheme-cleaneasy/theme/}}
\makeatother
\usepackage{beamerthemeCleanEasy}

% --- Essential packages ---
\usepackage{tikz}
\usepackage[american]{babel}
\usepackage{csquotes}
\usepackage[backend=biber, style=apa, sorting=nyt, doi=true, url=true]{biblatex}
\DeclareLanguageMapping{american}{american-apa}
\addbibresource{Oct22Meeting.bib}
\setbeamertemplate{bibliography item}{\insertbiblabel}

% --- Tables and figures ---
\usepackage{booktabs}
\usepackage{array}
\setbeamertemplate{caption}[numbered]
\setbeamerfont{caption}{size=\footnotesize}
\setbeamercolor{caption name}{fg=gray}

% --- Graphics paths ---
\graphicspath{{/Users/thomasmorton/subject-drop/analysis/paper_figures/main/}{/Users/thomasmorton/subject-drop/analysis/paper_figures/wide/}{/Users/thomasmorton/subject-drop/analysis/paper_figures/supplementary/}}

% --- Metadata ---
\title{Teaching AI to Learn Language Like Children Do}
\subtitle{A Detective Story About Grammar}
\author{Thomas Morton}
\date{October 22, 2025}
\institute{LemN Lab Meeting}

\begin{document}

% Title
\begin{frame}
  \titlepage
\end{frame}

% ============================================================================
% THE PROBLEM
% ============================================================================

\begin{frame}
  \frametitle{The Mystery}

  \begin{beamercolorbox}[wd=\textwidth,sep=0.4em,colsep=0pt]{block title}
    \textbf{How do children learn grammar without being taught?}
  \end{beamercolorbox}
  \begin{beamercolorbox}[wd=\textwidth,sep=0.3em,colsep=0pt]{block body}
    Children master complex language rules with minimal instruction
  \end{beamercolorbox}

  \vspace{1em}

  \textbf{Today's specific puzzle:}
  \begin{itemize}
    \item<2-> Some languages drop the subject: "Speaks Italian" (OK in Italian)
    \item<3-> Other languages require it: "She speaks Italian" (required in English)
    \item<4-> Children figure this out by age 3-4
    \item<5-> \textbf{But how?} Nobody explicitly teaches them this rule
  \end{itemize}
\end{frame}

\begin{frame}
  \frametitle{Why This Matters}

  \begin{columns}[T,onlytextwidth]
    \column{0.48\textwidth}
      \textbf{The Challenge:}
      \begin{itemize}
        \item Children only hear correct sentences
        \item Never told what's wrong
        \item Yet they learn perfectly
        \item Works across all languages
      \end{itemize}

    \column{0.48\textwidth}
      \textbf{The Stakes:}
      \begin{itemize}
        \item Understanding human cognition
        \item Building better AI
        \item Helping children with language disorders
        \item Designing language education
      \end{itemize}
  \end{columns}

  \vspace{1em}

  \onslide<2->{
  \begin{beamercolorbox}[wd=\textwidth,sep=0.3em,colsep=0pt]{block body}
    If we understand how children learn, we can build better learning systems
  \end{beamercolorbox}
  }
\end{frame}

% ============================================================================
% THE DEBATE
% ============================================================================

\begin{frame}
  \frametitle{40 Years of Debate}

  \textbf{Scientists have proposed different theories:}

  \vspace{1em}

  \begin{itemize}
    \item<2-> \textbf{Theory 1:} Children learn from specific words like "it" and "there"
    \item<3-> \textbf{Theory 2:} They detect patterns in word endings (-s, -ed, -ing)
    \item<4-> \textbf{Theory 3:} They count how often subjects appear
    \item<5-> \textbf{Theory 4:} They notice pronouns (I, you, he, she)
    \item<6-> \textbf{Theory 5:} They track articles (a, the)
  \end{itemize}

  \vspace{1em}

  \onslide<7->{
  \begin{beamercolorbox}[wd=\textwidth,sep=0.4em,colsep=0pt]{block title}
    \textbf{The Problem}
  \end{beamercolorbox}
  \begin{beamercolorbox}[wd=\textwidth,sep=0.3em,colsep=0pt]{block body}
    We can't test these theories on real children (unethical to manipulate their input)
  \end{beamercolorbox}
  }
\end{frame}

% ============================================================================
% THE SOLUTION
% ============================================================================

\begin{frame}
  \frametitle{The Solution: AI as a Test Subject}

  \begin{beamercolorbox}[wd=\textwidth,sep=0.4em,colsep=0pt]{block title}
    \textbf{Use Small Language Models as Stand-ins for Children}
  \end{beamercolorbox}
  \begin{beamercolorbox}[wd=\textwidth,sep=0.3em,colsep=0pt]{block body}
    We can control exactly what they learn from and track their progress
  \end{beamercolorbox}

  \vspace{1em}

  \textbf{Why this works:}
  \begin{itemize}
    \item<2-> Train on child-scale data (90M words = what a child hears)
    \item<3-> Remove specific evidence types to test theories
    \item<4-> Track learning over time like developmental psychology
    \item<5-> Run experiments impossible with humans
  \end{itemize}

  \vspace{1em}

  \onslide<6->{
  \textbf{The Approach:} Train multiple models, each missing different evidence
  }
\end{frame}

% ============================================================================
% THE EXPERIMENTS
% ============================================================================

\begin{frame}
  \frametitle{The Experiments}

  \begin{beamercolorbox}[wd=\textwidth,sep=0.3em,colsep=0pt]{block body}
    We trained 6 different AI models, each missing something different
  \end{beamercolorbox}

  \vspace{1em}

  \begin{tabular}{lll}
  \textbf{Model} & \textbf{What We Removed} & \textbf{Testing Which Theory} \\
  \hline
  Baseline & Nothing & Control \\
  Model 1 & Words like "it/there" & Expletive theory \\
  Model 2 & Article distinctions & Determiner theory \\
  Model 3 & All "a/the" & Article theory \\
  Model 4 & Word endings & Morphology theory \\
  Model 5 & Pronouns (I/you/he/she) & Direct evidence theory \\
  \end{tabular}

  \vspace{1em}

  \onslide<2->{
  \textbf{The Test:} Can they still learn that English requires subjects?
  }
\end{frame}

% ============================================================================
% KEY RESULTS WITH FIGURES
% ============================================================================

\begin{frame}{Result 1: Baseline Learning}
  \begin{columns}[T,onlytextwidth]
    \column{0.54\textwidth}
    \onslide<2->{
      \begin{figure}
      	\vspace{-1.5em}
		\includegraphics[width=1\linewidth]{model_baseline.pdf}
		\vspace{-1.5em}
		\caption{How the baseline model learns over time}
      \end{figure}
      }

    \column{0.44\textwidth}
      \raggedright
      \textbf{Key Finding:}
      \begin{itemize}
        \item<3-> Models start by dropping subjects (like Italian)
        \item<4-> Gradually learn English pattern
        \item<5-> Mirrors child development!
      \end{itemize}

      \vspace{0.5em}

      \onslide<6->{
      \textbf{Implication:}\\
      Default state might be to drop subjects
      }
  \end{columns}
\end{frame}

\begin{frame}{Result 2: The Surprise Winner}
  \begin{columns}[T,onlytextwidth]
    \column{0.54\textwidth}
    \onslide<2->{
      \begin{figure}
      	\vspace{-1.5em}
		\includegraphics[width=1\linewidth]{comparison_vs_baseline_overt_only_impoverish_determiners.pdf}
		\vspace{-1.5em}
		\caption{Effect of removing determiner distinctions}
      \end{figure}
      }

    \column{0.44\textwidth}
      \raggedright
      \textbf{Shocking Discovery:}
      \begin{itemize}
        \item<3-> Removing "a/the" distinctions causes \textbf{massive} delay
        \item<4-> Takes 5x longer to learn!
        \item<5-> Nobody predicted this
      \end{itemize}

      \vspace{0.5em}

      \onslide<6->{
      \textbf{What this means:}\\
      Articles might be a critical "shortcut" for learning grammar
      }
  \end{columns}
\end{frame}

\begin{frame}{Result 3: The Necessary Evidence}
  \begin{columns}[T,onlytextwidth]
    \column{0.54\textwidth}
    \onslide<2->{
      \begin{figure}
      	\vspace{-1.5em}
		\includegraphics[width=1\linewidth]{comparison_vs_baseline_overt_only_remove_subject_pronominals.pdf}
		\vspace{-1.5em}
		\caption{Effect of removing pronouns}
      \end{figure}
      }

    \column{0.44\textwidth}
      \raggedright
      \textbf{Critical Finding:}
      \begin{itemize}
        \item<3-> Without pronouns, learning barely happens
        \item<4-> Model stays at 50-50 (guessing)
        \item<5-> Pronouns are \textbf{necessary}
      \end{itemize}

      \vspace{0.5em}

      \onslide<6->{
      \textbf{Conclusion:}\\
      Direct evidence (seeing "I/you/he/she") is essential
      }
  \end{columns}
\end{frame}

\begin{frame}{The Big Picture}
  \begin{figure}
    \centering
    \vspace{-1em}
    \includegraphics[width=.85\linewidth]{all_models_comparison_log.pdf}
    \vspace{-1em}
    \caption{All models compared}
  \end{figure}

  \onslide<2->{
  \textbf{Evidence Hierarchy:}
  1. Pronouns (necessary) | 2. Articles (huge shortcut) | 3. Others (minor help)
  }
\end{frame}

% ============================================================================
% IMPLICATIONS
% ============================================================================

\begin{frame}
  \frametitle{What We Learned}

  \begin{beamercolorbox}[wd=\textwidth,sep=0.4em,colsep=0pt]{block title}
    \textbf{Major Discoveries}
  \end{beamercolorbox}

  \vspace{1em}

  \begin{enumerate}
    \item<2-> \textbf{Pronouns are essential}\\
    Without seeing "I/you/he/she", learning fails

    \vspace{0.5em}

    \item<3-> \textbf{Articles are shortcuts}\\
    "A/the" provide unexpected learning boost

    \vspace{0.5em}

    \item<4-> \textbf{Some theories are wrong}\\
    Word endings actually slow learning down!

    \vspace{0.5em}

    \item<5-> \textbf{Models mirror children}\\
    Start with wrong assumption, gradually correct it
  \end{enumerate}
\end{frame}

\begin{frame}
  \frametitle{Why This Matters}

  \textbf{For Science:}
  \begin{itemize}
    \item<2-> We can finally test competing theories
    \item<3-> Some 40-year-old theories are probably wrong
    \item<4-> Unexpected connections (articles → grammar learning)
  \end{itemize}

  \vspace{1em}

  \onslide<5->{
  \textbf{For AI:}
  \begin{itemize}
    \item<6-> Shows what data is actually important
    \item<7-> Could build more efficient learning systems
    \item<8-> Small models can reveal big insights
  \end{itemize}
  }

  \vspace{1em}

  \onslide<9->{
  \textbf{For Education:}
  \begin{itemize}
    \item<10-> Might inform language teaching methods
    \item<11-> Could help with language disorders
    \item<12-> Shows importance of pronouns and articles
  \end{itemize}
  }
\end{frame}

% ============================================================================
% NEXT STEPS
% ============================================================================

\begin{frame}
  \frametitle{Next Steps}

  \textbf{Immediate plans:}
  \begin{itemize}
    \item<2-> Test with other languages (Italian, Spanish, Chinese)
    \item<3-> Try different model architectures
    \item<4-> Look at bilingual learning
  \end{itemize}

  \vspace{1em}

  \onslide<5->{
  \textbf{Bigger questions:}
  \begin{itemize}
    \item<6-> Can we predict which children will struggle?
    \item<7-> How do bilingual children manage two systems?
    \item<8-> What other grammar rules work this way?
  \end{itemize}
  }

  \vspace{1em}

  \onslide<9->{
  \begin{beamercolorbox}[wd=\textwidth,sep=0.4em,colsep=0pt]{block title}
    \textbf{The Goal}
  \end{beamercolorbox}
  \begin{beamercolorbox}[wd=\textwidth,sep=0.3em,colsep=0pt]{block body}
    Build a complete theory of how grammar is learned from experience
  \end{beamercolorbox}
  }
\end{frame}

% ============================================================================
% CONCLUSION
% ============================================================================

\begin{frame}
  \frametitle{Take-Home Messages}

  \begin{enumerate}
    \item<2-> \textbf{AI can help us understand human learning}\\
    \vspace{0.3em}
    Small models trained on child-scale data reveal learning mechanisms

    \vspace{0.8em}

    \item<3-> \textbf{Not all evidence is equal}\\
    \vspace{0.3em}
    Pronouns: necessary | Articles: powerful shortcuts | Word endings: red herrings

    \vspace{0.8em}

    \item<4-> \textbf{Simple experiments can challenge old theories}\\
    \vspace{0.3em}
    40 years of debate, resolved with targeted ablation studies
  \end{enumerate}

  \vspace{1.5em}

  \onslide<5->{
  \begin{center}
    \Large
    \textbf{Questions?}
  \end{center}
  }
\end{frame}

% ============================================================================
% BACKUP SLIDES
% ============================================================================

\appendix

\begin{frame}
  \frametitle{Technical Details}

  \textbf{Model:} GPT-2 Small (124M parameters)

  \textbf{Data:} BabyLM corpus (90M words)
  \begin{itemize}
    \item Child-directed speech
    \item Children's books
    \item Simple Wikipedia
  \end{itemize}

  \textbf{Evaluation:} Minimal pairs
  \begin{itemize}
    \item "She walks" vs "*Walks"
    \item "It rains" vs "*Rains"
  \end{itemize}

  \textbf{Statistics:}
  \begin{itemize}
    \item 3 random seeds per condition
    \item Logistic mixed models
    \item Age of Acquisition metrics
  \end{itemize}
\end{frame}

\begin{frame}{Processing Effects}
  \begin{columns}[T,onlytextwidth]
    \column{0.54\textwidth}
      \begin{figure}
      	\vspace{-1.5em}
		\includegraphics[width=1\linewidth]{forest_form_baseline.pdf}
		\vspace{-1.5em}
		\caption{Effect of sentence complexity}
      \end{figure}

    \column{0.44\textwidth}
      \raggedright
      \textbf{Unexpected finding:}
      \begin{itemize}
        \item Negation increases subject use
        \item Opposite of human children
        \item Models don't "drop under pressure"
      \end{itemize}

      \vspace{0.5em}

      \textbf{Implication:}\\
      AI and humans might process difficulty differently
  \end{columns}
\end{frame}

\begin{frame}[allowframebreaks]{References}
  \printbibliography
\end{frame}

\end{document}