\documentclass{ucsdbrief}

\pretitle{Null Subject Parameter}
\title{Null Subjects in English and Italian}
\subtitle{A Cross-Linguistic Comparison}
\author{Linguistic Typology}
\date{\today}
\department{Department of Linguistics}
\affiliation{University of California San Diego}

\begin{document}
\maketitle

\chapter{The Null Subject Parameter}

\section{Italian: Consistent Null Subject Language (CNSL)}

\subsection{Minimal Pair in Italian}

\vspace{1em}

\textbf{(a) With Null Subject} \hfill \textcolor{green}{\checkmark Grammatical}

\vspace{0.5em}
\begin{center}
\begin{tabular}{llll}
\textit{---} & \textit{Parlo} & \textit{italiano.} & \\
--- & speak.\textsc{1sg} & Italian & \\
\multicolumn{4}{l}{\quad ``I speak Italian.''}
\end{tabular}
\end{center}

\vspace{2em}

\textbf{(b) With Overt Subject} \hfill \textcolor{green}{\checkmark Grammatical}

\vspace{0.5em}
\begin{center}
\begin{tabular}{llll}
\textit{Io} & \textit{parlo} & \textit{italiano.} & \\
I & speak.\textsc{1sg} & Italian & \\
\multicolumn{4}{l}{\quad ``I speak Italian.''}
\end{tabular}
\end{center}

\vspace{1em}

\alertbox{Italian permits null subjects under unconstrained conditions. Rich verbal inflection (\textit{parlo} = ``1sg speak'') identifies the missing subject.}

\newpage

\section{English: Non-Null Subject Language (NNSL)}

\subsection{Minimal Pair in English}

\vspace{1em}

\textbf{(a) With Null Subject} \hfill \textcolor{red}{* Ungrammatical}

\vspace{0.5em}
\begin{center}
\begin{tabular}{llll}
\textit{---} & \textit{Speak} & \textit{Italian.} & \\
--- & speak.\textsc{1sg} & Italian & \\
\multicolumn{4}{l}{\quad ``*I speak Italian.''}
\end{tabular}
\end{center}

\vspace{2em}

\textbf{(b) With Overt Subject} \hfill \textcolor{green}{\checkmark Grammatical}

\vspace{0.5em}
\begin{center}
\begin{tabular}{llll}
\textit{I} & \textit{speak} & \textit{Italian.} & \\
I & speak.\textsc{1sg} & Italian & \\
\multicolumn{4}{l}{\quad ``I speak Italian.''}
\end{tabular}
\end{center}

\vspace{1em}

\alertbox{English requires overt subjects in finite clauses. Verbal morphology (\textit{speak}) lacks rich agreement to identify the missing subject.}

\end{document}
