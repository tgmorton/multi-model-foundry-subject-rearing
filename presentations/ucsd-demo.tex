% Use the `standard` option to get 4x3 aspect ratio.
% Use the `noul` option if there is a conflict with the `\ul` command.
% Set theme with 'yellow', 'orange', 'red', 'blue', 'purple', or 'green'.
\documentclass{ucsdbrief}

% Remove overfull hbox warning markers
\setlength{\overfullrule}{0pt}

\pretitle{UCSD Presentation Template}
\title{Language Models and Human Language Processing}
\subtitle{Investigating Syntactic Learning in Neural Networks}
\author{Dr. Jane Smith \\ Dr. John Doe}
\date{October 16, 2025}
\department{Department of Cognitive Science}
\affiliation{University of California San Diego}
\footertext{UCSD Cognitive Science Symposium 2025}
% \copyrighted{2025}{UC San Diego}
% \contact{Custom contact information here}
% \extralogo{path/to/department/logo.png}
\motto{Fiat Lux - Let There Be Light}

% AI-friendly metadata (optional)
\topic{Language Models and Psycholinguistics}
\audience{Researchers, Graduate Students, Faculty}
\duration{45 minutes}
\keywords{language models, syntax, psycholinguistics, learning}

% Distribution and Control (optional for research use)
% \distributionA
% \cui{
%     Controlled By: UCSD \\
%     CUI Category(ies): PRVCY \\
%     Distribution: Limited \\
%     POC: Jane Smith, 858-555-1234}
% \banner{draft}

\begin{document}
\maketitle

\outline{
    \item Introduction and Features
    \item Research Methods and Data
    \item Code Examples
    \item Advanced Features
    \item AI-Friendly Features
    \item Research Presentation Patterns (NEW!)
}

\chapter{Introduction}

\section{Single Column}

Write your presentation like a normal \LaTeX\ file with a \verb|\maketitle|
command and \verb|\chapter| and \verb|\section| headings. The \verb|\maketitle|
contents are defined by the following macros:
\begin{center}
    \begin{tabular}{l@{\qquad}l@{\qquad}l}
        \verb|\pretitle| &
        \verb|\title| &
        \verb|\subtitle| \\
        \verb|\author| &
        \verb|\extralogo| &
        \verb|\motto|
    \end{tabular}
\end{center}
The \verb|\extralogo| command specifies an extra logo below the UCSD logo. The
\verb|\chapter| heading creates a slide with just the chapter name. The
\verb|\section| heading sets the title of a new slide. However, if no text
follows the section, no slide will be created. Text which does not fit on one
slide will flow onto the next slide automatically. To get 4-by-3 aspect ratio
slides, specify \verb|standard| as an option to the document class.

\section{Double Column}\twocolumn\raggedright

Use the \verb|\twocolumn| and \verb|\onecolumn| commands right after the section
heading to control the number of columns. Text will flow from the left column to
the right.
\begin{itemize}
    \item Syntactic priming studies
    \item Language model evaluation
    \item Cross-linguistic processing
    \item Sentence comprehension
    \item Neural network analysis
    \item Developmental trajectories
    \item Representational learning
    \item Computational modeling
    \item Behavioral experiments
    \item Transfer learning effects
    \item Attention mechanisms
    \item Gradient-based analysis
\end{itemize}
You can use \verb|\pagebreak| to force text onto the next column.

\section{Table of Research Areas}

You can create any variety of subdivisions on your slide by using the
\verb|tabular| environment.
\begin{center}
\begin{tabular}{C{0.25\textwidth}cC{0.25\textwidth}cC{0.25\textwidth}}
    \cellcolor{dark}\textcolor{white}{Linguistics} &&
    \cellcolor{dark}\textcolor{white}{Psychology} &&
    \cellcolor{dark}\textcolor{white}{Computation} \\
    Syntax && Cognition && Architectures \\
    Semantics && Processing && Training \\[1em]
    \cellcolor{dark}\textcolor{white}{Methods} &&
    \cellcolor{dark}\textcolor{white}{Theory} &&
    \cellcolor{dark}\textcolor{white}{Applications} \\
    Experiments && Frameworks && NLP \\
    Analysis && Models && Translation
\end{tabular}
\end{center}
The \verb|\cellcolor| command sets the background color of a table cell using
UCSD colors (\verb|dark| for navy blue, \verb|light| for gold).

\section{Quad Charts}

\quadchart{% Top-left content
    \textbf{Research Budget}\scriptsize\vspace{0.5cm}

    \begin{itemize}
        \item Budget: \$2,500,000
        \item Spent to Date: \$1,750,000
        \item Remaining: \$750,000
        \item Burn Rate: \$250,000/yr
        \item Projection: On track
    \end{itemize}
};
        }
    \end{tikzpicture}
    \end{center}
}{% Bottom-left content
    \textbf{Publications}\tiny\vspace{1em}

    \begin{center}
    \renewcommand{\arraystretch}{1.3}
    \begin{tabular}{@{}lrrr@{}}
        \hline
        \textbf{Type} & \textbf{Planned} &
            \textbf{Published} & \textbf{Remaining} \\
        \hline
        Journal Articles  & 15       & 10      & 5 \\
        Conference Papers & 8        & 6       & 2  \\
        Book Chapters     & 3        & 2       & 1  \\
        Reviews           & 5        & 3       & 2  \\
        Theses            & 4        & 4       & 0  \\
        \hline
        \textbf{Total}    & \textbf{35} &
            \textbf{25} & \textbf{10} \\
        \hline
    \end{tabular}
    \end{center}
}{% Bottom-right content
    \textbf{Research Impact}\tiny

    Our interdisciplinary research program has made significant contributions
    to understanding language acquisition and neural network learning. Key
    findings have been published in leading journals and have informed the
    development of more human-like language models. Continued support will
    enable groundbreaking discoveries in psycholinguistics and computational
    modeling of language.
}

\section{Centering}

\begin{Center}
    \Large Use the \texttt{Center} environment \\
    to center horizontally \emph{and} vertically.
\end{Center}

\chapter{Code Examples}

\section{Python}\onecolumn

Use the \verb|python| environment for Python code.
\begin{python}
def analyze_surprisal(model, sentences, threshold):
    """Analyze model surprisal for grammatical anomalies."""
    results = []
    for sent in sentences:
        surprisal = model.compute_surprisal(sent)
        perplexity = model.compute_perplexity(sent)
`\HL`        if surprisal > threshold or perplexity > 100:
            results.append({
                'sentence': sent,
                'anomaly_type': 'high_surprisal',
                'surprisal': surprisal,
                'perplexity': perplexity
            })
    return results
\end{python}

\section{MATLAB}

Use the \verb|matlab| environment for MATLAB code.
\begin{matlab}
function analyze_training_dynamics(epochs, loss, accuracy)
    % Analyze language model training dynamics
    % epochs, loss, accuracy are column vectors
    figure('Position', [100, 100, 800, 600]);

    subplot(1, 2, 1);
    plot(epochs, loss, 'LineWidth', 2);
    xlabel('Training Epoch');
    ylabel('Loss');
    title('Training Loss Curve');

    subplot(1, 2, 2);
    plot(epochs, accuracy, 'LineWidth', 2);
    xlabel('Training Epoch');
    ylabel('Accuracy');
    title('Model Accuracy');
end
\end{matlab}

% \section{R Language}

%Use the \verb|rlang| environment for R code.
%\begin{rlang}
%analyze_species_diversity <- function(data) {
%    # Calculate Shannon diversity index
%    species_counts <- table(data$species)
%    proportions <- species_counts / sum(species_counts)
%    diversity <- -sum(proportions * log(proportions))
%
%    return(list(
%        richness = length(species_counts),
%        shannon = diversity,
%        evenness = diversity / log(length(species_counts))
%    ))
%}
%\end{rlang}

\section{Pseudocode}

Use the \verb|pseudocode| environment for non-language-specific code.
\begin{pseudocode}
function compute_syntactic_priming($sentence$, $prime$, $model$)
    $surprisal_{null} \gets model.surprisal(sentence, prime_{null})$
    $surprisal_{overt} \gets model.surprisal(sentence, prime_{overt})$
    $priming\_effect \gets surprisal_{null} - surprisal_{overt}$

    if $|priming\_effect| > threshold$ then
        return $(priming\_effect, "significant")$
    else
        return $(priming\_effect, "not\_significant")$
    end if
end function
\end{pseudocode}

\chapter{Research Methods}

\section{Study Design}

Our research employs a multi-method approach to understanding language learning,
combining behavioral experiments, computational modeling, and neural network
analysis. Key methodologies include:

\subsection{Behavioral Experiments}
We conduct controlled experiments measuring human language processing using
eye-tracking, reaction time, and acceptability judgment tasks across various
linguistic constructions and populations.

\subsection{Computational Analysis}
Language model representations are analyzed using state-of-the-art techniques including:
\begin{itemize}
    \item Probing classifiers for syntactic representations
    \item Gradient-based attribution methods
    \item Attention pattern visualization
    \item Surprisal and perplexity metrics
\end{itemize}

\subsection{Neural Network Modeling}
We develop and train transformer-based language models to test hypotheses about
human language acquisition and predict behavioral patterns.

\section{Data Management}

All research data are archived following FAIR principles (Findable, Accessible,
Interoperable, and Reusable) and made publicly available through appropriate
repositories.

\chapter{Advanced Features}

\section{Callout Boxes}

The template now includes themed callout boxes for highlighting important information:

\alertbox{This is an alert box with the Note prefix. Use it to draw attention to important points or warnings.}

\vspace{1ex}

\infobox{This is an info box with subtle styling. Use it for supplementary information that readers should be aware of.}

\vspace{1ex}

\highlightbox{This is a highlight box for emphasizing key takeaways or important findings.}

\section{Theorem Environments}

\begin{theorem}[Surprisal-Frequency Relationship]
Let $M$ be a language model with word frequency distribution $f$ and surprisal function $S$. The surprisal of word $w$ satisfies:
$$S(w) \propto -\log f(w)$$
\end{theorem}

\begin{definition}[Cross-Entropy Loss]
The cross-entropy loss $\mathcal{L}$ for a language model over corpus $C$ is defined as:
$$\mathcal{L} = -\frac{1}{N}\sum_{i=1}^{N} \log P(w_i | w_{<i})$$
where $P(w_i | w_{<i})$ is the model's predicted probability of word $w_i$ given context.
\end{definition}

\begin{example}
For a uniform distribution over a 5-word vocabulary, $P(w_i) = 0.2$ for all $i$, yielding perplexity $2^H = 2^{2.32} = 5.0$.
\end{example}

\section{QR Codes}

Share your research data or paper easily with QR codes:

\begin{Center}
\qrlink[2cm]{https://doi.org/10.example/ucsd-language-models}

\small Link to research paper
\end{Center}

% You can create blank pages for placing full-page graphics or text with the
% `\blankpage` command.
\blankpage
% You can place images and text at arbitrary locations with the `\pos` command.
\pos{0pt}{0pt}{\tikz{
    \newcounter{density}\setcounter{density}{10}
    \def\mainColor{light}
    \path[coordinate] (0,0) coordinate(A)
        ++(\paperwidth,0) coordinate(B)
        ++(0,-\paperheight) coordinate(C)
        ++(-\paperwidth,0) coordinate(D);
    \fill[\mainColor!\thedensity]
        (A) -- (B) -- (C) -- (D) -- cycle;
    \foreach \x in {1,...,18}{%
        \pgfmathsetcounter{density}{\thedensity+5}
        \setcounter{density}{\thedensity}
        \path[coordinate] coordinate(X) at (A){};
        \path[coordinate] (A)
            -- (B) coordinate[pos=0.15](A)
            -- (C) coordinate[pos=0.15](B)
            -- (D) coordinate[pos=0.15](C)
            -- (X) coordinate[pos=0.15](D);
        \draw[\mainColor!80!black, fill=\mainColor!\thedensity]
            (A) -- (B) -- (C) -- (D) -- cycle;
    }
}}
\pos[8cm]{0.125\textwidth}{2cm}{
    \raggedright\Large\color{dark}
    \textbf{Blank Page Example}

    \normalsize
    Use the \texttt{\textbackslash blankpage} command to create pages with
    custom graphics and layouts.
}
\pos[8cm]{0.125\textwidth}{6cm}{
    \raggedright\Large\color{dark}
    \textbf{Arbitrary Positioning}

    \normalsize
    The \texttt{\textbackslash pos} command allows precise placement of
    content anywhere on the page using absolute coordinates.
}

\chapter{AI-Friendly Features}

\section{Structured Content Blocks}

The template includes structured content blocks that help AI agents understand
the purpose and organization of slide content:

\slidegoal{Demonstrate how structured content blocks improve clarity and
organization for both human readers and AI agents.}

\context{These commands provide semantic meaning to slide content, making it
easier for AI to generate, parse, and understand presentation materials.}

Key benefits:
\begin{itemize}
    \item Clear communication of slide objectives
    \item Contextual information for better understanding
    \item Highlighted key takeaways
    \item Proper source attribution
\end{itemize}

\takeaway{Structured content blocks make presentations more maintainable and
AI-friendly while improving clarity for human audiences.}

\source{UCSD Presentation Template Documentation}

\keypoint{Key Points}{Machine Learning transforms marine research through
automated species classification and ecosystem modeling.}

\section{Timeline Example}

\timeline{Evolution of Marine Research Technology}{
    \timelineitem{1960s}{Development of early underwater cameras and basic
    sonar systems}
    \timelineitem{1980s}{Introduction of remotely operated vehicles (ROVs) for
    deep sea exploration}
    \timelineitem{2000s}{Autonomous underwater vehicles (AUVs) with advanced
    sensors}
    \timelineitem{2020s}{AI-powered analysis of underwater imagery and
    environmental DNA sequencing}
}

\splitslide{Split Slide Layout}{
    \textbf{Left Column}

    This is the left side of a split slide. Perfect for comparisons or
    presenting two related concepts side by side.

    \begin{itemize}
        \item Traditional methods
        \item Manual data collection
        \item Limited coverage
        \item Time-intensive analysis
    \end{itemize}
}{
    \textbf{Right Column}

    This is the right side of the split. The content automatically flows into
    two equal columns.

    \begin{itemize}
        \item Modern AI approaches
        \item Automated sensors
        \item Large-scale monitoring
        \item Real-time insights
    \end{itemize}
}

\comparison{Research Approaches}{Traditional Field Work}{
    \begin{itemize}
        \item Ship-based sampling
        \item Manual species identification
        \item Limited temporal coverage
        \item High cost per sample
        \item Expert taxonomist required
    \end{itemize}
}{Automated Monitoring}{
    \begin{itemize}
        \item Continuous sensor deployment
        \item AI-powered image analysis
        \item Long-term time series
        \item Cost-effective at scale
        \item Accessible to more researchers
    \end{itemize}
}

\problemsolution{Challenges in Ocean Monitoring}{
    \textbf{The Problem:}

    Monitoring vast ocean ecosystems requires extensive resources, yet climate
    change impacts demand comprehensive, real-time data. Traditional methods
    cannot scale to meet these needs, creating critical gaps in our
    understanding of marine ecosystem health.

    Key challenges:
    \begin{itemize}
        \item Insufficient spatial coverage
        \item Limited temporal resolution
        \item High operational costs
        \item Data processing bottlenecks
    \end{itemize}
}{
    \textbf{Our Solution:}

    Deploy a network of autonomous sensors combined with AI-powered analysis to
    provide continuous, large-scale ocean monitoring at reduced cost.

    Approach:
    \begin{itemize}
        \item Low-cost sensor arrays
        \item Satellite data integration
        \item Machine learning models
        \item Cloud-based processing
        \item Open data sharing
    \end{itemize}
}

\beforeafter{Impact of Conservation Efforts}{
    \textbf{Status in 2010:}

    \begin{itemize}
        \item Coral cover: 15\%
        \item Fish species: 45
        \item Water quality: Poor
        \item Tourist impact: High
        \item Protection: Minimal
    \end{itemize}

    Significant ecosystem degradation with declining biodiversity and
    deteriorating water quality.
}{
    \textbf{Status in 2024:}

    \begin{itemize}
        \item Coral cover: 42\%
        \item Fish species: 87
        \item Water quality: Excellent
        \item Tourist impact: Managed
        \item Protection: Marine reserve
    \end{itemize}

    Remarkable ecosystem recovery following implementation of conservation
    measures and community engagement.
}

\proscons{AI in Marine Research}{
    \item Rapid analysis of large datasets
    \item Consistent, objective classification
    \item 24/7 monitoring capability
    \item Scales to global coverage
    \item Reduces human bias
    \item Cost-effective long-term
}{
    \item Requires training data
    \item May miss novel species
    \item Initial setup costs
    \item Needs validation
    \item Limited contextual understanding
    \item Dependent on sensor quality
}

\section{Icon Items}

\iconitem{\textbf{1}}{First, deploy autonomous sensors across study sites to
establish baseline environmental measurements.}

\iconitem{\textbf{2}}{Next, collect and aggregate data through satellite
uplinks for centralized processing.}

\iconitem{\textbf{3}}{Then, apply machine learning models to identify patterns
and anomalies in the data streams.}

\iconitem{\textbf{4}}{Finally, generate actionable insights for conservation
and policy decisions.}

\section{Grid Layout Example}

\gridlayout{Four Research Areas}{
    \textbf{Biodiversity}

    Species distribution mapping using eDNA and imaging
}{
    \textbf{Climate Impact}

    Temperature and pH monitoring for acidification studies
}{
    \textbf{Conservation}

    Protected area effectiveness assessment
}{
    \textbf{Human Impact}

    Pollution tracking and coastal development effects
}

\dataslide{Research Output Metrics}{
    Below is sample data visualization showing our research productivity:

    \vspace{2ex}

    \simpletable{Publication Metrics by Year}{l|r|r|r}{
        \textbf{Year} & \textbf{Papers} & \textbf{Citations} & \textbf{h-index} \\
        \hline
        2022 & 18 & 342 & 12 \\
        2023 & 24 & 478 & 15 \\
        2024 & 31 & 621 & 18 \\
    }

    \vspace{2ex}

    Note how structured data commands make it easy for AI to parse and
    generate tables.
}

\section{Simple Bar Graph}

The \texttt{\textbackslash bargraph} command creates quick visualizations:

\bargraph{Quarterly Research Funding (\$K)}{Q1,Q2,Q3,Q4}{(Q1,250) (Q2,380) (Q3,420) (Q4,510)}

This makes it easy for AI agents to generate data visualizations from simple
numerical inputs.

\chapter{Research Presentation Patterns}

\statements{Research Motivation}{
    I want to investigate \textbf{how language models learn} syntactic
    generalizations from linguistic input.
}{
    I seek to use \textbf{controlled training experiments} as a method to
    \textbf{isolate evidence sources} that drive language acquisition.
}{
    I chose to focus on \textbf{null-subject constraints} as a case study in
    this investigation.
}

\researchquestions{Study Overview}{
    \begin{itemize}
        \item Do small language models acquire null-subject constraints in a
        human-like way?
        \item What linguistic evidence types are necessary for learning the
        constraint?
        \item Can we predict acquisition trajectories from training data
        properties?
        \item How do bilingual models show cross-linguistic transfer in null
        subject learning?
    \end{itemize}
}

\methodslide{Experimental Design}{
    Controlled rearing study with systematic ablation of linguistic evidence
}{
  	\\ \textbf{Training Conditions:}
    \begin{itemize}
        \item 6 experimental conditions (5 ablations + baseline)
        \item 90M word developmentally-plausible corpus
        \item Transformer architecture (GPT-2 small)
        \item Training period: 5 epochs with dense checkpointing
    \end{itemize}

    \vspace{1ex}

    \textbf{Evaluation Methods:}
    \begin{itemize}
        \item Surprisal-based preference measurements
        \item Minimal pair comparisons (null vs. overt subjects)
        \item Age of acquisition analysis
        \item Cross-linguistic transfer testing
    \end{itemize}
}

\designmatrix{Experimental Conditions}{
    {\small
    \begin{tabular}{p{0.28\columnwidth}|p{0.32\columnwidth}|p{0.28\columnwidth}}
        \textbf{Condition} & \textbf{Ablation} & \textbf{Prediction} \\
        \hline
        Baseline & None & Adult-like preference \\
        Remove Expletives & No \textit{it}/\textit{there} & Delayed acquisition \\
        Impoverish Determiners & \textit{a}/\textit{the} $\rightarrow$ DET & Slower learning \\
        Remove Articles & No \textit{a}/\textit{the} & Moderate delay \\
        Lemmatize Verbs & No inflections & Faster learning \\
        Remove Pronouns & No subject pronouns & Near-chance \\
        \hline
    \end{tabular}
    }

    \vspace{2ex}

    {\small \textbf{Measurement:} Overt subject preference measured at
    log-spaced checkpoints throughout training}
}

\section{Results Example with Mock Figure}

% Note: In real usage, replace this with an actual figure path
\begin{minipage}[t]{0.52\columnwidth}
    \begin{center}
        % Placeholder for figure - in real use: \includegraphics[width=\linewidth]{your_figure.pdf}
        \begin{tikzpicture}
            \draw[fill=light!30] (0,0) rectangle (6,4);
            \node at (3,2) {\textbf{[Your Figure Here]}};
            \node[align=center] at (3,1.2) {\small Overt Preference\\by Training Epoch};
        \end{tikzpicture}
        \captionof{figure}{Learning curves showing baseline model (blue)
        acquires overt preference faster than ablated models (red)}
    \end{center}
\end{minipage}\hfill
\begin{minipage}[t]{0.45\columnwidth}
    \raggedright
    \vspace{-8em}
    \textbf{Key Findings:}
    \begin{itemize}
        \item Baseline achieved \textbf{AoA at checkpoint 727} (95\% CI
        [664, 791], p$<$.001)
        \item Subject pronouns were the strongest predictor of successful
        acquisition (r=0.89)
        \item Models with determiners learned more consistently across runs
        \item All models showed initial null-subject preference before reversal
    \end{itemize}
\end{minipage}

\section{Cross-Model Comparison Matrix}

\begin{center}
    \begin{tabular}{l|c|c|c}
        \textbf{Model} & \textbf{Overt Pref (\%)} & \textbf{AoA (epochs)} & \textbf{$\Delta$AoA} \\
        \hline
        Baseline & 69.6 & 727 & -- \\
        Remove Expletives & 67.2 & 767 & +39* \\
        Impoverish Determiners & 64.7 & 3400 & +2672*** \\
        Remove Articles & 66.4 & 807 & +80** \\
        Lemmatize Verbs & 62.2 & 705 & -22* \\
        Remove Pronouns & 56.1 & 774 & +47* \\
        \hline
        \textbf{Mean Direct Evidence} & \textbf{68.4} & \textbf{747} & -- \\
        \textbf{Mean Indirect Evidence} & \textbf{64.4} & \textbf{1637} & -- \\
    \end{tabular}
\end{center}

\vspace{2ex}

Statistical analysis reveals significant differences between evidence types
across all acquisition metrics (* p$<$.05, ** p$<$.01, *** p$<$.001).

\comparison{Theoretical Accounts}{Competence-Based Learning}{
    \begin{itemize}
        \item Grammatical parameter setting
        \item Direct evidence from input
        \item Gradual statistical learning
        \item Adult-like end state
        \item Evidence-driven acquisition
    \end{itemize}
}{Performance-Based Errors}{
    \begin{itemize}
        \item Processing load effects
        \item Resource limitations
        \item Context-dependent omissions
        \item Developmental trajectory
        \item Cognitive constraints
    \end{itemize}
}

\problemsolution{Poverty of the Stimulus Challenge}{
    \textbf{The Problem:}

    Children acquire complex syntactic constraints from limited positive
    evidence, with little to no negative feedback. Current theoretical accounts
    struggle to explain how learners converge on correct grammars from
    ambiguous input.

    \textbf{Critical Issues:}
    \begin{itemize}
        \item Indirect evidence may be insufficient for learning
        \item Direct negative evidence is rare or absent
        \item Multiple hypotheses consistent with input
        \item Cross-linguistic variation requires parameter setting
    \end{itemize}
}{
    \textbf{Proposed Solution:}

    Use controlled rearing experiments with language models to test which
    evidence types are sufficient for constraint acquisition.

    \textbf{Research Plan:}
    \begin{itemize}
        \item Train models on developmentally-plausible corpora
        \item Systematically ablate specific evidence sources
        \item Measure acquisition trajectories and end-state competence
        \item Test cross-linguistic transfer and priming effects
        \item Compare model learning to child developmental patterns
    \end{itemize}
}

\section{Implications for Acquisition Theory}

\takeaway{Subject pronouns provide critical direct evidence for null-subject
constraints, while determiner systems facilitate learning through indirect cues.
Models show human-like developmental patterns, validating the use of controlled
rearing as a method for testing acquisition theories.}

\vspace{2ex}

\textbf{Theoretical Implications:}
\begin{enumerate}
    \item Hyams' direct evidence account receives strong support from pronoun
    ablation effects
    \item Duguine's determiner hypothesis explains accelerated learning with
    rich nominal morphology
    \item Yang's expletive-based account shows minimal effect in our models
    \item Processing accounts fail to predict model behavior under complexity
    manipulations
    \item Evidence hierarchy emerges: pronouns $>$ determiners $>$ verbal morphology $>$ expletives
\end{enumerate}

\source{Morton et al. (2025), Cognitive Science}

\closing
\end{document}
