% For draft mode without transitions, uncomment the next line:
%\documentclass[aspectratio=169,handout]{beamer}
% For presentation mode with transitions, use this line:
\documentclass[aspectratio=169]{beamer}

% --- Theme (manual import if you have the .sty locally) ---
% Either:
\usepackage{beamerthemeCleanEasy}
% or, if installed via TeX Live/MiKTeX:
% \usetheme{CleanEasy}

% --- Language & APA bibliography ---
\usepackage[american]{babel}
\usepackage{csquotes}
\usepackage[backend=biber, style=apa, sorting=nyt, doi=true, url=true]{biblatex}
\DeclareLanguageMapping{american}{american-apa}
\addbibresource{Jul10.bib} % <-- point to your .bib file

% --- Table packages ---
\usepackage{booktabs}
\usepackage{array}
\usepackage{makecell}  % for \thead command
\usepackage{pifont}    % for \ding symbols

% --- Linguistic examples ---
% Using standard LaTeX for examples to avoid package conflicts

% --- Figure/caption tuning ---
\setbeamertemplate{caption}[numbered]
\setbeamerfont{caption}{size=\footnotesize}
\setbeamercolor{caption name}{fg=gray}

% --- Graphics search paths (shorten your include paths) ---
\graphicspath{{../analysis/paper_figures/main/}{../analysis/paper_figures/wide/}{../analysis/paper_figures/supplementary/}}

\title{Small Language Models as a tool to investigate the acquisition of the Null-Subject constraint}
\author{Thomas Morton}
\date{August 21, 2025}

\begin{document}

% Content

\begin{frame}
  \titlepage
\end{frame}

\begin{frame}
  \centering
  \Large
  
  \vspace{1em}
  
  I want to investigate \textbf{how humans learn syntactic generalizations} from linguistic evidence.
  
  \vspace{0.8em}
  
  I seek to use \textbf{large language models as candidate models} of a language learner that we can \textbf{intervene on and investigate} the representations of.
  
  \vspace{0.8em}
  
  I chose to use the \textbf{acquisition and representation of null subjects} as a case study in this investigation.
  
\end{frame}

% Introduction: Theoretical Foundation
\begin{frame}
  \frametitle{The Null-Subject Constraint in Language Acquisition}
  
  \begin{beamercolorbox}[wd=\textwidth,sep=0.4em,colsep=0pt]{block title}
    \textbf{Cross-Linguistic Variation}
  \end{beamercolorbox}
  \begin{beamercolorbox}[wd=\textwidth,sep=0.3em,colsep=0pt]{block body}
    Languages differ in whether they allow phonologically null subjects
  \end{beamercolorbox}
  
  \vspace{0.8em}
  
  \begin{columns}[T,onlytextwidth]
    \column{0.48\textwidth}
      \textbf{English (Non-pro-drop):}
      \begin{itemize}
        \item \textit{She/*$\emptyset$ runs}
        \item \textit{It/*$\emptyset$ rains}
        \item Overt subjects required
      \end{itemize}
      
    \column{0.48\textwidth}
      \textbf{Italian (Pro-drop):}
      \begin{itemize}
        \item \textit{(Lei) corre} '(She) runs'
        \item \textit{$\emptyset$/*Ci piove} '(It) rains'
        \item Rich verbal agreement
      \end{itemize}
  \end{columns}
  
  \vspace{0.8em}
  
  \textbf{Acquisition Challenge \parencite{Hyams1986-ae,Rizzi1994-tm} :}
  \begin{itemize}
    \item Children show early null subjects across languages
    \item Must learn when overt subjects are required vs. optional
  \end{itemize}
\end{frame}

% Poverty of Stimulus and Learnability
\begin{frame}
  \frametitle{The Poverty of Stimulus Problem}
  
  \begin{beamercolorbox}[wd=\textwidth,sep=0.4em,colsep=0pt]{block title}
    \textbf{Learnability Challenge}
  \end{beamercolorbox}
  \begin{beamercolorbox}[wd=\textwidth,sep=0.3em,colsep=0pt]{block body}
    How do children learn from positive evidence alone? \parencite{Hyams1993-zk}
  \end{beamercolorbox}
  
  \vspace{0.8em}
  
  \textbf{The Problem:}
  \begin{itemize}
    \item<1-> Children rarely receive negative evidence (*"$\emptyset$ runs" is ungrammatical)
    \item<2-> Yet they successfully acquire language-specific constraints
    \item<3-> \textbf{Poverty of Stimulus:} Input seems insufficient for learning \parencite{Chomsky1959-ni}
  \end{itemize}
  
  \vspace{0.8em}
  
  \onslide<4->{
  \textbf{Competing Accounts:}
  \begin{itemize}
    \item<4-> \textbf{Indirect distributional evidence:} Orthogonal grammatical information guides linguistic generalizations \parencite{Yang2004-wk}
    \item<5-> \textbf{Direct evidence:} Overt forms signal constraints \parencite{Hyams1993-zk}
  \end{itemize}
  }
\end{frame}

% Large Language Models as Learners
\begin{frame}
  \frametitle{Large Language Models as Candidate Learners}
  
%  \begin{beamercolorbox}[wd=\textwidth,sep=0.4em,colsep=0pt]{block title}
%    \textbf{A New Test Case}
%  \end{beamercolorbox}
%  \begin{beamercolorbox}[wd=\textwidth,sep=0.3em,colsep=0pt]{block body}
%    Do LLMs learn the null-subject constraint like child learners?
%  \end{beamercolorbox}
  
  \vspace{0.8em}
  
  \textbf{Key Advantages:}
  \begin{itemize}
    \item<1-> \textbf{Controlled input:} Manipulate training data systematically
    \item<2-> \textbf{Developmental tracking:} Observe learning over time
    \item<3-> \textbf{Multiple architectures:} Test different learning mechanisms
  \end{itemize}
  
  \vspace{0.8em}
  
  \onslide<4->{
  \textbf{Recent Evidence \parencite{Warstadt2020-yx,Hu2020-bb}:}
  \begin{itemize}
    \item<4-> LLMs acquire many syntactic generalizations
    \item<5-> Show human-like developmental trajectories
    \item<6-> BUT: Trained on massive datasets ($>>$child input)
  \end{itemize}
  }
  
  \vspace{0.8em}
  
  \onslide<7->{
  \begin{beamercolorbox}[wd=\textwidth,sep=0.3em,colsep=0pt]{block body}
    \textbf{This Work:} Train models on developmentally-plausible corpora to test questions of sufficiency of the language input.
  \end{beamercolorbox}
  }
\end{frame}

% Contravariance Principle
\begin{frame}
  \frametitle{The Contravariance Principle}
  
  \begin{beamercolorbox}[wd=\textwidth,sep=0.4em,colsep=0pt]{block title}
    \textbf{Why Do Models and Humans Converge? \parencite{Cao2021-xi}}
  \end{beamercolorbox}
  \begin{beamercolorbox}[wd=\textwidth,sep=0.3em,colsep=0pt]{block body}
    Hard computational problems constrain possible solutions
  \end{beamercolorbox}
  
  \vspace{1em}
  
  \textbf{Core Insight:}
  \begin{itemize}
    \item<1-> \textbf{Hard problems} require satisfying multiple competing constraints
    \item<2-> \textbf{Few viable solutions} exist under such constraints
    \item<3-> \textbf{Different systems} solving the same hard problem converge
  \end{itemize}
  
  \vspace{1em}
  
  \onslide<4->{
  \textbf{Applied to Language:}
  \begin{itemize}
    \item<4-> Null-subject acquisition involves multiple competing constraints
    \item<5-> Flexible learners (models and humans) should converge to similar solutions
    \item<6-> Convergence toward the simplest viable explanation
  \end{itemize}
  }
\end{frame}

% Planonic Representation Hypothesis
\begin{frame}
  \frametitle{The Planonic Representation Hypothesis}
  
  \begin{beamercolorbox}[wd=\textwidth,sep=0.4em,colsep=0pt]{block title}
    \textbf{Linking Theory: Internal Representations $\leftrightarrow$ Linguistic Competence}
  \end{beamercolorbox}
  \begin{beamercolorbox}[wd=\textwidth,sep=0.3em,colsep=0pt]{block body}
    Abstract syntactic knowledge should be observable in model representations
  \end{beamercolorbox}
  
  \vspace{1em}
  
  \textbf{Key Claims:}
  \begin{itemize}
    \item<1-> \textbf{Structural priming} reveals abstract grammatical representations \parencite{Bock1986-ej}
    \item<2-> \textbf{Cross-linguistic effects} suggest language-independent structure \parencite{Michaelov2023-bg,Arnett2025-kv}
    \item<3-> \textbf{Null elements} can be positively represented \parencite{Momma2018-dl}
  \end{itemize}
  
  \vspace{1em}
  
  \onslide<4->{
  \textbf{Methodological Innovation:}
  \begin{itemize}
    \item<4-> Use structural priming to probe model competence
    \item<5-> Test representations beyond surface performance
    \item<6-> Bridge psycholinguistics and computational modeling
  \end{itemize}
  }
\end{frame}

% Predictions and Applications
\begin{frame}
  \frametitle{Predictions for Null-Subject Learning}
  
  \textbf{Theoretical Prediction:}
  \begin{itemize}
    \item<1-> Models solving the hard null-subject problem should converge to human-like representations
    \item<2-> The Model's generalizations should effect language learning in the same way that it effects human multilingual learning
    \item<3-> Abstract representations should transcend surface linguistic differences
  \end{itemize}
  
  \vspace{1em}
  
  \onslide<4->{
  \textbf{Empirical Tests:}
  \begin{itemize}
    \item<1-> \textbf{Chapter 1:} Do models learn null-subject constraints like humans?
    \item<2-> \textbf{Chapter 2:} Do bilingual models show human-like transfer effects?
    \item<3-> \textbf{Chapter 3:} Do models exhibit cross-linguistic structural priming?
  \end{itemize}
  }
 
\end{frame}

\begin{frame}{Contents}
  % Chapter 1
  \begin{beamercolorbox}[wd=\textwidth,sep=0.4em,colsep=0pt]{block title}
    \textbf{Chapter 1: A controlled rearing study of the null-subject constraint in English}
  \end{beamercolorbox}
  \begin{beamercolorbox}[wd=\textwidth,sep=0.3em,colsep=0pt]{block body}
    {\footnotesize Investigate the contribution of individual sources of evidence in the acquisition of the null-subject constraint by performing ablative experiments on the datasets}
  \end{beamercolorbox}
  
  \vspace{0.5em}
  
  % Chapter 2
  \begin{beamercolorbox}[wd=\textwidth,sep=0.4em,colsep=0pt]{block title alerted}
    \textbf{Chapter 2: Transfer effects in bilingual acquisition of the null-subject constraint}
  \end{beamercolorbox}
  \begin{beamercolorbox}[wd=\textwidth,sep=0.3em,colsep=0pt]{block body alerted}
    {\footnotesize Investigate the cross-language transfer effects of learning competing null-subject generalizations in sequential language learning}
  \end{beamercolorbox}
  
  \vspace{0.5em}
  
  % Chapter 3
  \begin{beamercolorbox}[wd=\textwidth,sep=0.4em,colsep=0pt]{block title example}
    \textbf{Chapter 3: The syntactic priming of null-subjects cross-linguistically}
  \end{beamercolorbox}
  \begin{beamercolorbox}[wd=\textwidth,sep=0.3em,colsep=0pt]{block body example}
    {\footnotesize Investigate models abstract representations using syntactic priming effects as a measure.}
  \end{beamercolorbox}
\end{frame}


% ==================================================================
% CHAPTER 1: A controlled rearing study of the null-subject constraint in English
% ==================================================================

% Chapter 1 overview
\begin{frame}
  \frametitle{Chapter 1: A controlled rearing study of the null-subject constraint in English}
  \begin{beamercolorbox}[wd=\textwidth,sep=0.4em,colsep=0pt]{block body}
    \textbf{Research Questions:}
    \begin{itemize}
    	\item<1-> Do English Small Language Models learn the null-subject constraint in a human-like way?

    	\item<2-> What kind of linguistic information contributes to the learning of that constraint?

    	\item<3-> Do model's show human-like behavior in contexts with higher processing demands?
    \end{itemize}
  \end{beamercolorbox}
  
  \vspace{0.8em}
  
  \onslide<4->{
  \textbf{Approach:}
  \begin{itemize}
    \item<4-> Controlled rearing experiments with ablated training data
    \item<5-> Systematic removal of specific linguistic evidence types
    \item<6-> Manipulation of evaluation stimuli to examine contextual processing effects
  \end{itemize}
  }
  
  \vspace{0.5em}
\end{frame}


% The null-subject constraint
\begin{frame}
  \frametitle{The Null-Subject Constraint in English}
  
  \begin{beamercolorbox}[wd=\textwidth,sep=0.4em,colsep=0pt]{block title}
    \textbf{What is the null-subject constraint?}
  \end{beamercolorbox}
  \begin{beamercolorbox}[wd=\textwidth,sep=0.3em,colsep=0pt]{block body}
    English requires overt subjects in finite clauses (unlike Spanish, Italian, etc.)
  \end{beamercolorbox}
  
  \vspace{0.8em}
  
  \textbf{Adult English Constraint:}
  \begin{itemize}
    \item[\textcolor{red}{\textbf{*}}] \emph{$\emptyset$ Finished the book} (ungrammatical)
    \item[\textcolor{green}{\checkmark}] \emph{She finished the book} (grammatical)
  \end{itemize}
  
  \vspace{0.5em}
  
  \textbf{Child Null-Subject Use Examples:}
  \begin{itemize}
    \item \textit{Shake hands.}
    \item \textit{Turn light off.}
    \item \textit{Want go get it.}
    \item \textit{Show mommy that.}
    \item \textit{Now making muffins.}
  \end{itemize}
  
  
  
  
 \end{frame}

% Accounts of Subject Drop
\begin{frame}
  \frametitle{Performance vs. Competence Accounts}
  
  \begin{beamercolorbox}[wd=\textwidth,sep=0.4em,colsep=0pt]{block title}
    \textbf{Why Do Children Drop Subjects?}
  \end{beamercolorbox}
  \begin{beamercolorbox}[wd=\textwidth,sep=0.3em,colsep=0pt]{block body}
    Two competing explanations for child null-subject patterns
  \end{beamercolorbox}
  
  \vspace{0.8em}
  
  \begin{columns}[T,onlytextwidth]
    \column{0.48\textwidth}
      \textbf{Performance Account \parencite{Bloom1970-xr,Bloom1990-tz}:}
      \begin{itemize}
        \item Children drop subjects under processing load
        \item Cognitive resource limitations
        \item More drops with negation, longer sentences
        \item Subject/object asymmetry due to planning
      \end{itemize}
      
    \column{0.48\textwidth}
      \textbf{Competence Account \parencite{Hyams1986-ae,Hyams1993-zk}:}
      \begin{itemize}
        \item Children initially set null-subject parameter
        \item Must learn overt subject requirement
        \item Grammatical learning, not performance
        \item Direct evidence from overt pronouns
      \end{itemize}
  \end{columns}
  
  \vspace{0.8em}
  
  \textbf{Key Debate:} Processing limitation vs. grammatical parameter setting
\end{frame}

\begin{frame}
  \frametitle{Specific Theoretical Accounts}
  
  \textbf{Yang's Variational Learning \parencite{Yang2003-fn,Yang2004-wk}:}
  \begin{itemize}
    \item<1-> \textbf{Expletives} are crucial unambiguous evidence
    \item<2-> Probabilistic grammar competition 
    \item<3-> Rarity of expletives explains slow English acquisition
  \end{itemize}
  
  \vspace{0.8em}
  
  \onslide<4->{
  \textbf{Hyams' Triggering Theory \parencite{Hyams1993-zk}:}
  \begin{itemize}
    \item<4-> \textbf{Non-uniform verbal morphology} triggers parameter reset
    \item<5-> Italian: uniformly rich, Mandarin: uniformly poor $\rightarrow$ null subjects
    \item<6-> English: inconsistent system $\rightarrow$ overt subjects required
  \end{itemize}
  }
  
  \vspace{0.8em}
  
  \onslide<7->{
  \textbf{Duguine's Inverse Approach \parencite{Duguine2017-fr}:}
  \begin{itemize}
    \item<7-> \textbf{Rich determiners + weak agreement} $\rightarrow$ overt subjects
    \item<8-> Nominal domain provides indirect evidence
    \item<9-> Focus shifts from verbal to determiner system
  \end{itemize}`	
  }
\end{frame}

\begin{frame}
  \frametitle{Experimental Design Rationale}
  
  \begin{beamercolorbox}[wd=\textwidth,sep=0.4em,colsep=0pt]{block title}
    \textbf{Testing Causal Contributions}
  \end{beamercolorbox}
  \begin{beamercolorbox}[wd=\textwidth,sep=0.3em,colsep=0pt]{block body}
    Each experiment targets specific theoretical predictions
  \end{beamercolorbox}
  
  \vspace{0.8em}
  
  \textbf{Theory-Experiment Mapping:}
  \begin{itemize}
    \item<1-> \textbf{Exp 1 - Remove Expletives:} Tests Yang's prediction about expletive evidence
    \item<2-> \textbf{Exp 2 - Impoverish Determiners:} Tests Duguine's rich determiner hypothesis  
    \item<3-> \textbf{Exp 3 - Remove Articles:} Further tests determiner system importance
    \item<4-> \textbf{Exp 4 - Lemmatize Verbs:} Tests Hyams' verbal morphology prediction
    \item<5-> \textbf{Exp 5 - Remove Pronouns:} Tests direct vs. indirect evidence accounts
  \end{itemize}
  
\end{frame}

%% Sources of evidence
%\begin{frame}
%  \frametitle{Sources of Evidence for the Null-Subject Constraint}
%  
%  \textbf{What linguistic evidence might inform acquisition?}
%  
%  \vspace{0.8em}
%  
%  \begin{columns}[T,onlytextwidth]
%    \column{0.48\textwidth}
%      \begin{beamercolorbox}[wd=\textwidth,sep=0.3em,colsep=0pt]{block title}
%        \textbf{Positive Evidence}
%      \end{beamercolorbox}
%      \begin{beamercolorbox}[wd=\textwidth,sep=0.2em,colsep=0pt]{block body}
%        \begin{itemize}
%          \item Expletive subjects: \emph{It rains}
%          \item Pronominal subjects: \emph{She left}
%          \item Full DP subjects: \emph{The cat slept}
%        \end{itemize}
%      \end{beamercolorbox}
%      
%    \column{0.48\textwidth}
%      \begin{beamercolorbox}[wd=\textwidth,sep=0.3em,colsep=0pt]{block title}
%        \textbf{Morphological Cues}
%      \end{beamercolorbox}
%      \begin{beamercolorbox}[wd=\textwidth,sep=0.2em,colsep=0pt]{block body}
%        \begin{itemize}
%          \item Verbal agreement: \emph{-s} marking
%          \item Determiner system: \emph{a/the}
%          \item Case marking (limited in English)
%        \end{itemize}
%      \end{beamercolorbox}
%  \end{columns}
%  
%  \vspace{0.8em}
%  
%  \begin{beamercolorbox}[wd=\textwidth,sep=0.3em,colsep=0pt]{block body}
%    \textbf{Research Question:} Which of these evidence sources are \emph{necessary} vs. \emph{sufficient} for acquiring the constraint?
%  \end{beamercolorbox}
%\end{frame}

% Experimental design
\begin{frame}
  \frametitle{Experimental Design: Controlled Rearing}
  
  \begin{beamercolorbox}[wd=\textwidth,sep=0.4em,colsep=0pt]{block title}
    \textbf{Controlled Rearing Paradigm}
  \end{beamercolorbox}
  \begin{beamercolorbox}[wd=\textwidth,sep=0.3em,colsep=0pt]{block body}
    Train models on systematically modified datasets to isolate evidence contributions
  \end{beamercolorbox}
  
  \vspace{0.8em}
  
  \textbf{Ablation Experiments:}
  \begin{enumerate}
    \item[0] \textbf{Baseline:} Full training corpus
    \item[1] \textbf{Remove Expletives:} No \emph{it/there} expletive constructions
    \item[2] \textbf{Impoverish Determiners:} Reduce \emph{a/the} to \emph{DET}
    \item[3] \textbf{Remove Articles:} No \emph{a/the} entirely
    \item[4] \textbf{Lemmatize Verbs:} Remove \emph{-s/-ed/-ing} morphology  
    \item[5] \textbf{Remove Subject Pronominals:} No \emph{I/you/he/she/it/we/they}
  \end{enumerate}
  
  \vspace{0.5em}
  
  \textbf{Evaluation:} Null vs. overt subject preferences in controlled contexts
\end{frame}

% Experimental conditions
%\begin{frame}
%  \frametitle{Experimental Conditions: Ablation Studies}
%  
%  \begin{columns}[T,onlytextwidth]
%    \column{0.48\textwidth}
%      \begin{beamercolorbox}[wd=\textwidth,sep=0.3em,colsep=0pt]{block title}
%        \textbf{Experiment 0: Baseline}
%      \end{beamercolorbox}
%      \begin{beamercolorbox}[wd=\textwidth,sep=0.2em,colsep=0pt]{block body}
%        {\footnotesize Control condition with complete linguistic evidence}
%      \end{beamercolorbox}
%      
%      \vspace{0.4em}
%      
%      \begin{beamercolorbox}[wd=\textwidth,sep=0.3em,colsep=0pt]{block title}
%        \textbf{Experiment 1: Remove Expletives}
%      \end{beamercolorbox}
%      \begin{beamercolorbox}[wd=\textwidth,sep=0.2em,colsep=0pt]{block body}
%        {\footnotesize Eliminate dummy subjects to test direct evidence importance}
%      \end{beamercolorbox}
%      
%      \vspace{0.4em}
%      
%      \begin{beamercolorbox}[wd=\textwidth,sep=0.3em,colsep=0pt]{block title}
%        \textbf{Experiment 2: Impoverish Determiners}
%      \end{beamercolorbox}
%      \begin{beamercolorbox}[wd=\textwidth,sep=0.2em,colsep=0pt]{block body}
%        {\footnotesize Reduce morphological richness while preserving structure}
%      \end{beamercolorbox}
%      
%    \column{0.48\textwidth}
%      \begin{beamercolorbox}[wd=\textwidth,sep=0.3em,colsep=0pt]{block title}
%        \textbf{Experiment 3: Remove Articles}
%      \end{beamercolorbox}
%      \begin{beamercolorbox}[wd=\textwidth,sep=0.2em,colsep=0pt]{block body}
%        {\footnotesize Create article-less environment like some languages}
%      \end{beamercolorbox}
%      
%      \vspace{0.4em}
%      
%      \begin{beamercolorbox}[wd=\textwidth,sep=0.3em,colsep=0pt]{block title}
%        \textbf{Experiment 4: Lemmatize Verbs}
%      \end{beamercolorbox}
%      \begin{beamercolorbox}[wd=\textwidth,sep=0.2em,colsep=0pt]{block body}
%        {\footnotesize Eliminate verbal morphology and agreement markers}
%      \end{beamercolorbox}
%      
%      \vspace{0.4em}
%      
%      \begin{beamercolorbox}[wd=\textwidth,sep=0.3em,colsep=0pt]{block title}
%        \textbf{Experiment 5: Remove Subject Pronouns}
%      \end{beamercolorbox}
%      \begin{beamercolorbox}[wd=\textwidth,sep=0.2em,colsep=0pt]{block body}
%        {\footnotesize Eliminate direct evidence for obligatory subjects}
%      \end{beamercolorbox}
%  \end{columns}
%  
%  \vspace{0.5em}
%  
%\end{frame}

% Examples

% MEASUREMENT AND ANALYSIS

\begin{frame}{Measures and Analysis: Overview}
  \begin{block}{Data and Coding}
    \begin{itemize}
      \item Binary outcome: \texttt{correct} = 1 when \textit{null} $<$ \textit{overt} surprisal
      \item Factors: \texttt{model}, \texttt{form\_type}, \texttt{item\_group}, \texttt{form}
      \item Training progress: $\log_{10}(\texttt{checkpoint}+1)$
      \item Baseline condition as reference level
    \end{itemize}
  \end{block}
  
  \begin{block}{Outcome Definition}
    \begin{itemize}
      \item Minimal pairs: \textit{null} vs. \textit{overt} subject realization
      \item Binary response $Y \in \{0,1\}$ encodes preference
      \item End-state: \textit{overt} preference (probability scale)
      \item Acquisition-time: \textit{null} preference
    \end{itemize}
  \end{block}
\end{frame}

\begin{frame}{Logistic Models: Learning Curves and Splines}
  \begin{block}{GLMMs}
    $$\operatorname{logit}\Pr(Y=1) = \beta_0 + \operatorname{ns}\bigl(\log_{10}(t+1),\,k\bigr) + u_i$$
    \begin{itemize}
      \item Natural spline over log-checkpoint, complexity $k$
      \item Random intercept: $u_{\texttt{item}} \sim \mathcal{N}(0,\sigma^2)$
      \item Spline selection: AIC over $K \in \{3,\ldots,7\}$
    \end{itemize}
  \end{block}
  
  \begin{block}{Training Progress}
    \begin{itemize}
      \item Log$_{10}$ scale: $\{0, 10, 100, 1\mathrm{K}, 10\mathrm{K}\}$
      \item Reflects neural network log-learning dynamics
      \item Uniform checkpointing across conditions
    \end{itemize}
  \end{block}
\end{frame}

\begin{frame}{Age of Acquisition (AoA) Analysis}
  \begin{block}{$t_{50}$ (Chance-Level Acquisition)}
    \begin{itemize}
      \item Last crossing of 0.50 after burn-in ($\geq 100$ checkpoints)
      \item Linear interpolation between fitted points
      \item Right-censored if no crossing
      \item Bootstrap 95\% CIs ($n=500$)
    \end{itemize}
  \end{block}
  
  \begin{block}{AoA$_{1/2}$ (Halfway-to-Asymptote)}
    \begin{itemize}
      \item End-state $p_{\infty}$ from last 10\% of training
      \item Threshold: $\theta = (p_{\infty} + 0.5)/2$
      \item First post-burn-in crossing of $\theta$
      \item Between-model $\Delta$AoA$_{1/2}$ via paired bootstrap
    \end{itemize}
  \end{block}
\end{frame}

\begin{frame}{Materials: BabyLM Dataset}
  \textbf{Training Corpus}
  \begin{itemize}
    \item 90M word corpus designed for human-sized models
    \item Linguistically diverse with child-directed speech
    \item Models linguistic input of 10-14 year old child
    \item 10M word held-out test set
    \item 10M word ablation replacement set
  \end{itemize}
  
  \vspace{1em}
  \textbf{Dataset Composition}
  \begin{itemize}
    \item CHILDES (child-directed speech): 29M words
    \item Project Gutenberg (children's stories): 26M words  
    \item OpenSubtitles (movie subtitles): 20M words
    \item Simple English Wikipedia: 15M words
    \item BNC dialogue + Switchboard: 9M words
  \end{itemize}
\end{frame}

\begin{frame}{Evaluation Stimuli: Null vs. Overt Subjects}
  \begin{columns}[T,onlytextwidth]
    \column{0.5\textwidth}
    \textbf{Core Contrasts (English non-pro-drop)}
    \begin{itemize}
      \item \textbf{Person/Number}: Anna finished. She/*$\emptyset$ thinks... 
      \item \textbf{Control}: Maria convinced her brother $\emptyset$/*him to leave
      \item \textbf{Expletives}: *$\emptyset$/It seems that students passed
      \item \textbf{Topic shift}: Anna called Mark and *$\emptyset$/he refused
    \end{itemize}
    
    \column{0.5\textwidth}
    \textbf{Minimal Pairs Design}
    \begin{itemize}
      \item Sentences differ only in subject realization
      \item Lexical and contextual content held constant
      \item Testsfamilies
      \item Evaluates grammatical vs. processing accounts
    \end{itemize}
  \end{columns}
\end{frame}

\begin{frame}{Processing Manipulations}
  \textbf{Context Complexity (Bloom 1990)}
  \begin{enumerate}
    \item \textbf{Simple}: The dog barked. He/*$\emptyset$ scared...
    \item \textbf{Long NPs}: The large brown dog with red collar barked...
    \item \textbf{Embedded}: The dog that lived in the house...
  \end{enumerate}
  
  \vspace{1em}
  \textbf{Negation Effects (Bloom 1970)}
  \begin{enumerate}
    \item \textbf{Target negation}: She/*$\emptyset$ doesn't think...
    \item \textbf{Context negation}: Anna didn't finish. She/*$\emptyset$ thinks...
    \item \textbf{Double negation}: Both context and target negated
  \end{enumerate}
  
  \vspace{0.5em}
  \textit{Tests processing load effects on subject drop preferences}
\end{frame}

% RESULTS

\begin{frame}{Baseline Model -- Training Curves}
  \begin{columns}[T,onlytextwidth]
    \column{0.50\textwidth}
    \onslide<2->{
      \begin{figure}
      	\vspace{-1.5em}
		\includegraphics[width=1.1\linewidth]{analysis/paper_figures/main/model_baseline.pdf}
		}
		\vspace{-1.5em}
		\caption{Model preference for null and overt evaluation stimuli over training, training steps transformed to log-scale to reflect model log-learning dynamics for Experiment 0 - Baseline} 
      \end{figure}
      
    \column{0.46\textwidth}
      \raggedright
      \begin{itemize}
        \item<3-> Age of Acquisition analysis revealed that baseline achieved \textbf{AoA at checkpoint 727} \\{\small (95\% CI [664, 791])}. 
        \item<4-> A \textbf{63.4\% preference for null subjects over first epoch} \\ {\small (95\% CI [62.7, 64.1], \textit{p}$<$.001)}
        \item<5-> a \textbf{69.6\% preference for overt subjects in the last two epochs of training} \\ {\small (95\% CI [66.5\%, 72.5\%], \textit{p} $<$ .001)}
      \end{itemize}
      \medskip
  \end{columns}
\end{frame}

\begin{frame}{Exp 1: `Remove Expletives' -- Training Curves}
  \begin{columns}[T,onlytextwidth]
    \column{0.54\textwidth}
    \onslide<2->{
      \begin{figure}
      	\vspace{-1.5em}
		\includegraphics[width=1\linewidth]{analysis/paper_figures/wide/comparison_vs_baseline_overt_only_remove_expletives.pdf}
		}
		\vspace{-1.5em}
		\caption{Model overt preference over training, training steps transformed to log-scale to reflect model log-learning dynamics comparing Experiment 0 and Experiment 1.} 
      \end{figure}
      
    \column{0.46\textwidth}
      \raggedright
      \begin{itemize}
        \item<3-> Age of Acquisition analysis revealed that baseline achieved \textbf{AoA at checkpoint 767} \\{\small (95\% CI [709, 821])}. 
        \begin{itemize}
        	\item Which is significantly later than the baseline model {\small ($\Delta$AoA = 39 epochs, 95\% CI [24, 55], p$<$.001)}
        \end{itemize}
        \medskip
        \item<4-> Start-performance and end-performance did not significantly differ from base model.
      \end{itemize}
      \medskip
  \end{columns}
\end{frame}

\begin{frame}{Exp 2: `Impoverish Determiners' -- Training Curves}
  \begin{columns}[T,onlytextwidth]
    \column{0.54\textwidth}
    \onslide<2->{
      \begin{figure}
      	\vspace{-1.5em}
		\includegraphics[width=1\linewidth]{analysis/paper_figures/wide/comparison_vs_baseline_overt_only_impoverish_determiners.pdf}
		}
		\vspace{-1.5em}
		\caption{Model overt preference over training, training steps transformed to log-scale to reflect model log-learning dynamics comparing Experiment 0 and Experiment 1.} 
      \end{figure}
      
    \column{0.46\textwidth}
      \raggedright
      \begin{itemize}
        \item<3-> Age of Acquisition analysis revealed that baseline achieved \textbf{AoA at checkpoint 3400} \\{\small (95\% CI [3307, 3499])}. 
        \begin{itemize}
        	\item Which is significantly later than baseline {\small ($\Delta$AoA = 2672 epochs, 95\% CI [2620, 2724], p$<$.001)}
        \end{itemize}
        \medskip
        \item<4-> The model had a significant, but smaller preference for null subjects by the end of the first epoch.
        \item<5-> By the end of the final two epochs, it has the strongest preference of all models for overt subjects
      \end{itemize}
      \medskip
  \end{columns}
\end{frame}

\begin{frame}{Exp 3: `Remove Articles' -- Training Curves}
  \begin{columns}[T,onlytextwidth]
    \column{0.54\textwidth}
    \onslide<2->{
      \begin{figure}
      	\vspace{-1.5em}
		\includegraphics[width=1\linewidth]{analysis/paper_figures/wide/comparison_vs_baseline_overt_only_remove_articles.pdf}
		}
		\vspace{-1.5em}
		\caption{Model overt preference over training, training steps transformed to log-scale to reflect model log-learning dynamics comparing Experiment 0 and Experiment 3.} 
      \end{figure}
      
    \column{0.46\textwidth}
      \raggedright
      \begin{itemize}
        \item<3-> Age of Acquisition analysis revealed that this model achieved \textbf{AoA at checkpoint 807} \\{\small (95\% CI [758, 861])}. 
        \begin{itemize}
        	\item Which is significantly later than baseline {\small ($\Delta$AoA = 80 epochs, 95\% CI [81, 108], p$<$.001)}
        \end{itemize}
        \medskip
        \item<4-> Shows significantly stronger null preference in first epoch (71.7\%) compared to baseline.
        \item<5-> End-state overt preference (68.2\%) is significantly lower than baseline model.
      \end{itemize}
      \medskip
  \end{columns}
\end{frame}

\begin{frame}{Exp 4: `Lemmatize Verbs' -- Training Curves}
  \begin{columns}[T,onlytextwidth]
    \column{0.54\textwidth}
    \onslide<2->{
      \begin{figure}
      	\vspace{-1.5em}
		\includegraphics[width=1\linewidth]{analysis/paper_figures/wide/comparison_vs_baseline_overt_only_lemmatize_verbs.pdf}
		}
		\vspace{-1.5em}
		\caption{Model overt preference over training, training steps transformed to log-scale to reflect model log-learning dynamics comparing Experiment 0 and Experiment 4.} 
      \end{figure}
      
    \column{0.46\textwidth}
      \raggedright
      \begin{itemize}
        \item<3-> Age of Acquisition analysis revealed that this model achieved \textbf{AoA at checkpoint 705} \\{\small (95\% CI [660, 748])}. 
        \begin{itemize}
        	\item Which is significantly \textbf{earlier} than baseline {\small ($\Delta$AoA = -22 epochs, 95\% CI [-43, -1.65], p = .034)}
        \end{itemize}
        \medskip
        \item<4-> Fastest acquisition among all interventions.
      \end{itemize}
      \medskip
  \end{columns}
\end{frame}

\begin{frame}{Exp 5: `Remove Subject Pronominals' -- Training Curves}
  \begin{columns}[T,onlytextwidth]
    \column{0.54\textwidth}
    \onslide<2->{
      \begin{figure}
      	\vspace{-1.5em}
		\includegraphics[width=1\linewidth]{analysis/paper_figures/wide/comparison_vs_baseline_overt_only_remove_subject_pronominals.pdf}
		}
		\vspace{-1.5em}
		\caption{Model overt preference over training, training steps transformed to log-scale to reflect model log-learning dynamics comparing Experiment 0 and Experiment 5.} 
      \end{figure}
      
    \column{0.46\textwidth}
      \raggedright
      \begin{itemize}
        \item<3-> Age of Acquisition analysis revealed that this model achieved \textbf{AoA at checkpoint 774} \\{\small (95\% CI [706, $>$5000])}. 
        \begin{itemize}
        	\item Slightly later than baseline {\small ($\Delta$AoA = 47 epochs, p$<$.05)}
        \end{itemize}
        \medskip
        \item<4-> Weakest overall overt preference (54.4\%) among all models.
      \end{itemize}
      \medskip
  \end{columns}
\end{frame}

\begin{frame}{Cross-Model Comparison}
%  Prior work suggests a modest baseline effect \parencite{einstein1905}. 
%  We follow the reporting recommendations of \textcite{doe2020guidelines}.
%  \medskip

  \begin{figure}
    \centering
    \vspace{-1em}
    \caption{Cross-model comparison of null subject acquisition trajectories (log scale)}
    \vspace{-.5em}
    \includegraphics[width=.9\linewidth]{analysis/paper_figures/wide/all_models_comparison_log.pdf}
  \end{figure}
\end{frame}

% ==================================================================
% PROCESSING ACCOUNT RESULTS
% ==================================================================

% Processing account overview
\begin{frame}
  \frametitle{Processing Account: Predicted vs. Observed}
  
  \begin{beamercolorbox}[wd=\textwidth,sep=0.4em,colsep=0pt]{block title}
    \textbf{Bloom's Processing Account Prediction}
  \end{beamercolorbox}
  \begin{beamercolorbox}[wd=\textwidth,sep=0.3em,colsep=0pt]{block body}
    Under increased processing load, children should drop subjects MORE frequently
  \end{beamercolorbox}
  
  \vspace{0.8em}
  
  \begin{columns}[T,onlytextwidth]
    \column{0.48\textwidth}
      \textbf{Processing Manipulations:}
      \begin{itemize}
        \item Long noun phrases
        \item Embedded relative clauses  
        \item Negation contexts
        \item Target vs. context negation
      \end{itemize}
      
    \column{0.48\textwidth}
%      \begin{beamercolorbox}[wd=\textwidth,sep=0.3em,colsep=0pt]{block title}
%        \textbf{Our Findings}
%      \end{beamercolorbox}
%      \begin{beamercolorbox}[wd=\textwidth,sep=0.2em,colsep=0pt]{block body}
%        {\footnotesize Models show INCREASED overt subject use under complexity - opposite of prediction}
%      \end{beamercolorbox}
  \end{columns}
  
  \vspace{1em}
  
  \begin{beamercolorbox}[wd=\textwidth,sep=0.3em,colsep=0pt]{block body}
    \textbf{Implication:} Do LLMs omit more subjects in contexts with heavy processing load?
  \end{beamercolorbox}
\end{frame}

\begin{frame}{Exp 5: `Remove Subject Pronominals' -- Training Curves}
  \begin{columns}[T,onlytextwidth]
    \column{0.54\textwidth}
    \onslide<2->{
      \begin{figure}
      	\vspace{-1.5em}
		\includegraphics[width=1\linewidth]{analysis/paper_figures/supplementary/forest_form_baseline.pdf}
		}
		\vspace{-1.5em}
		\caption{Model overt preference over training, training steps transformed to log-scale to reflect model log-learning dynamics comparing Experiment 0 and Experiment 5.} 
      \end{figure}
      
    \column{0.46\textwidth}
      \raggedright
      \begin{itemize}
        \item<3-> Negation shows the strongest influence on models' choice for overt subjects.
        \item<4-> The model under these contexts show increased overt preference
        \item<5-> This is counter to the reported human pattern of higher null subject use in negation contexts.
      \end{itemize}
      \medskip
  \end{columns}
\end{frame}

% /Users/thomasmorton/subject-drop/analysis/paper_figures/supplementary/forest_form_baseline.pdf

% Processing results by manipulation
\begin{frame}
  \frametitle{Processing Effects Across All Models}
  
  \centering
  \resizebox{0.95\textwidth}{!}{%
    \input{../analysis/tables/latex_tables/forms_vs_default_checkmarks}%
  }
  
  \vspace{0.4em}
  \addtocounter{table}{1}
  {\footnotesize \textbf{Table \thetable:} Syntactic forms showing significant deviation from default performance by experimental model}
  

  \begin{itemize}
    \item<2-> \textbf{Target/Both negation:} Universally increases overt preference
    \item<3-> \textbf{Complex syntax:} Largely does not increase overt preference  
    \item<4-> \textbf{Context negation:} No effect across models
  \end{itemize}
\end{frame}

% Theoretical implications
\begin{frame}
  \frametitle{Implications for Processing Theories}
  
  \begin{columns}[T,onlytextwidth]
    \column{0.48\textwidth}
      \begin{beamercolorbox}[wd=\textwidth,sep=0.3em,colsep=0pt]{block title}
        \textbf{Traditional View}
      \end{beamercolorbox}
      \begin{beamercolorbox}[wd=\textwidth,sep=0.2em,colsep=0pt]{block body}
        \begin{itemize}
          \item Processing load $\rightarrow$ omit subjects
          \item "Good enough" processing
          \item Resource limitation effects
        \end{itemize}
      \end{beamercolorbox}
      
    \column{0.48\textwidth}
      \begin{beamercolorbox}[wd=\textwidth,sep=0.3em,colsep=0pt]{block title}
        \textbf{Model Behavior}
      \end{beamercolorbox}
      \begin{beamercolorbox}[wd=\textwidth,sep=0.2em,colsep=0pt]{block body}
        \begin{itemize}
          \item Processing load $\rightarrow$ \emph{insert} subjects
          \item More explicit under complexity
          \item Robust to context effects
        \end{itemize}
      \end{beamercolorbox}
  \end{columns}
  
  \vspace{0.8em}
  
  \textbf{Possible Explanations:}
  \begin{itemize}
    \item Models and children process complexity fundamentally differently
    \item Processing accounts may not fully explain child null subject errors
    \item Need empirical validation of processing effects in human production
  \end{itemize}
\end{frame}

% ==================================================================
% DISCUSSION: MODEL LEARNING BEHAVIOR
% ==================================================================

% Early null subject stage
\begin{frame}
  \frametitle{Universal Early Null Subject Stage}
  
  \begin{beamercolorbox}[wd=\textwidth,sep=0.4em,colsep=0pt]{block title}
    \textbf{Surprising Finding: All Models Show Initial Null Subject Preference}
  \end{beamercolorbox}
  \begin{beamercolorbox}[wd=\textwidth,sep=0.3em,colsep=0pt]{block body}
    Despite English being overt-subject, ALL models prefer null subjects early in training
  \end{beamercolorbox}
  
  \vspace{0.8em}
  
  \textbf{Theoretical Implications:}
  \begin{itemize}
    \item<2-> Consistent with some child acquisition patterns (null-first accounts)
    \item<3-> Contradicts Bloom's prediction that English children default to overt subjects
    \item<4-> Could reflect model architecture bias OR environmental evidence
    \item<5-> Most models acquire English-like preferences after approx. 1.5 epochs. 
  \end{itemize}
  
  \vspace{0.5em}
  
  \onslide<6->{
  \begin{beamercolorbox}[wd=\textwidth,sep=0.3em,colsep=0pt]{block body}
    \textbf{Question:} Is this a learning bias or evidence from the input environment?
  \end{beamercolorbox}
  }
\end{frame}

% Evidence hierarchy and grokking
\begin{frame}
  \frametitle{Evidence Types: Shortcuts vs. Deep Learning}
  
  \begin{columns}[T,onlytextwidth]
    \column{0.48\textwidth}
      \begin{beamercolorbox}[wd=\textwidth,sep=0.3em,colsep=0pt]{block title}
        \textbf{Direct Evidence}
      \end{beamercolorbox}
      \begin{beamercolorbox}[wd=\textwidth,sep=0.2em,colsep=0pt]{block body}
        \begin{itemize}
          \item \textbf{Subject pronouns:} Critical
          \item Remove pronouns $\rightarrow$ near-chance performance
          \item Supports Hyams' direct evidence account
        \end{itemize}
      \end{beamercolorbox}
      
    \column{0.48\textwidth}
      \begin{beamercolorbox}[wd=\textwidth,sep=0.3em,colsep=0pt]{block title}
        \textbf{Indirect Evidence}
      \end{beamercolorbox}
      \begin{beamercolorbox}[wd=\textwidth,sep=0.2em,colsep=0pt]{block body}
        \begin{itemize}
          \item \textbf{Determiners:} Provide "shortcuts"
          \item \textbf{Verbal morphology:} Affects final strength
          \item \textbf{Expletives:} Minimal effect
        \end{itemize}
      \end{beamercolorbox}
  \end{columns}
  
  \vspace{0.8em}
  
  \begin{beamercolorbox}[wd=\textwidth,sep=0.4em,colsep=0pt]{block title}
    \textbf{Grokking Hypothesis}
  \end{beamercolorbox}
  \begin{beamercolorbox}[wd=\textwidth,sep=0.3em,colsep=0pt]{block body}
    Removing shortcuts (determiners) forces slower but potentially more robust generalization
  \end{beamercolorbox}
\end{frame}

% Person hierarchy
%\begin{frame}
%  \frametitle{Person Hierarchies: Universal Grammatical Patterns}
%  
%  \textbf{Robust Pattern Across All Models:}
%  
%  \vspace{0.5em}
%  
%  \begin{center}
%    \Large
%    \textbf{1st person} $>$ \textbf{2nd person} $\approx$ \textbf{3rd person}
%    
%    \vspace{0.3em}
%    
%    {\footnotesize (More likely to produce overt subjects)}
%  \end{center}
%  
%  %\vspace{0.8em}
%  
%  \textbf{Cross-Linguistic Evidence:}
%  \begin{itemize}
%    \item Mirrors person hierarchies in pro-drop languages (Spanish, Italian)
%    \item Suggests models capture universal grammatical principles
%    \item 3rd person singular behaves more like determiners in some conditions
%  \end{itemize}
%  
%  \vspace{0.5em}
%  
%  \textbf{Theoretical Support:}
%  \begin{itemize}
%    \item Carminati (2005): Person-based processing differences
%    \item Bonet (1991): 3rd person pronouns as determiners
%  \end{itemize}
%\end{frame}

% Theoretical implications
\begin{frame}
  \frametitle{Broader Theoretical Implications}
  
  \begin{beamercolorbox}[wd=\textwidth,sep=0.4em,colsep=0pt]{block title}
    \textbf{What This Study Challenges}
  \end{beamercolorbox}
  \begin{beamercolorbox}[wd=\textwidth,sep=0.3em,colsep=0pt]{block body}
    Multiple acquisition theories need revision based on these findings
  \end{beamercolorbox}
  
  \vspace{0.8em}
  
  \begin{columns}[T,onlytextwidth]
    \column{0.48\textwidth}
      \textbf{Challenges:}
      \begin{itemize}
        \item Yang's variational learning
        \item Processing-based accounts
        \item Simple parameter-setting
      \end{itemize}
      
    \column{0.48\textwidth}
      \textbf{Supports:}
      \begin{itemize}
        \item Hyams' direct evidence account
        \item Universal grammar constraints
        \item Gradual, evidence-based learning
      \end{itemize}
  \end{columns}
  
  \vspace{0.8em}
  
  \begin{beamercolorbox}[wd=\textwidth,sep=0.3em,colsep=0pt]{block body}
    \textbf{Future Work:} Test these patterns with human participants and cross-linguistic data
  \end{beamercolorbox}
\end{frame}

% /Users/thomasmorton/subject-drop/analysis/paper_figures/wide/comparison_vs_baseline_overt_only_impoverish_determiners.pdf

% /Users/thomasmorton/subject-drop/analysis/paper_figures/wide/comparison_vs_baseline_overt_only_remove_expletives.pdf

%\begin{frame}{Forest plot overview}
%  \begin{columns}[T,onlytextwidth]
%    \column{0.42\textwidth}
%      \begin{figure}
%      \onslide<2->{
%      	\vspace{-1.5em}
%        \includegraphics[height=0.6\textheight]{forest_item_group_baseline.pdf}}
%        \vspace{-1.5em}
%        \caption{Model preferences for overt subjects by processing manipulation at final checkpoint for Experiment 0 - Baseline}
%      \end{figure}
%    \column{0.4\textwidth}
%      \raggedright
%      \begin{itemize}
%        \item Effects cluster near zero; two groups stand out \parencite{smith2021effects}.
%      \end{itemize}
%  \end{columns}
%\end{frame}



%% ------------------------------------------------------------------
%% Example: Model preferences table
%% ------------------------------------------------------------------
%\begin{frame}{Model Preference Summary}
%  \centering
%  \begin{table}
%  	\vspace{-2em}
%    \include{analysis/tables/latex_tables/simple_aoa_table}
%    \caption{Null vs. Overt Subject Preferences by Model}
%  \end{table}
%  \footnotesize All models show overt preference, with Remove Subject Pronominals showing the smallest bias.
%\end{frame}
%
%% ------------------------------------------------------------------
%% Example: Acquisition timing table
%% ------------------------------------------------------------------
%\begin{frame}{Age of Acquisition Comparison}
%  \begin{columns}[T,onlytextwidth]
%    \column{0.55\textwidth}
%      \resizebox{0.9\textwidth}{!}{%
%        \input{analysis/tables/latex_tables/simple_aoa_table}%
%      }
%      
%      \vspace{0.3em}
%      {\footnotesize \textbf{Table:} Crossover Points by Model}
%    \column{0.4\textwidth}
%      \raggedright
%      \textbf{Key findings:}
%      \begin{itemize}
%        \item Lemmatize Verbs achieves earliest acquisition
%        \item Impoverish Determiners shows dramatically delayed acquisition
%        \item Most interventions cluster around 700-800 checkpoints
%      \end{itemize}
%  \end{columns}
%\end{frame}
%
%% ------------------------------------------------------------------
%% Example: cite in text while showing a full-width figure
%% ------------------------------------------------------------------
%\begin{frame}{Baseline model at a glance}
%  Prior work suggests a modest baseline effect \parencite{einstein1905}. 
%  We follow the reporting recommendations of \textcite{doe2020guidelines}.
%  \medskip
%
%  \begin{figure}
%    \centering
%    \includegraphics[width=.6\linewidth]{model_baseline.pdf}
%    \caption{Model baseline (vector PDF keeps labels crisp on projector).}
%  \end{figure}
%\end{frame}
%
%% ------------------------------------------------------------------
%% Example: tall figure LEFT, commentary RIGHT, with citations
%% ------------------------------------------------------------------
%\begin{frame}{Forest plot overview \hfill \normalsize Notes}
%  \begin{columns}[T,onlytextwidth]
%    \column{0.42\textwidth}
%      \begin{figure}
%        \includegraphics[height=0.6\textheight]{forest_item_group_baseline.pdf}
%        \caption{Item-group baseline.}
%      \end{figure}
%    \column{0.4\textwidth}
%      \raggedright
%      \textbf{Key takeaways}
%      \begin{itemize}
%        \item Effects cluster near zero; two groups stand out \parencite{smith2021effects}.
%      \end{itemize}
%      \medskip
%      \textbf{Reading tips}
%      \begin{itemize}
%        \item Align axis/units with other slides for fast comparison.
%        \item Legends: show once; avoid duplicates to reduce clutter.
%      \end{itemize}
%  \end{columns}
%\end{frame}
%
%% ------------------------------------------------------------------
%% Example: side-by-side comparison with citations in bullets
%% ------------------------------------------------------------------
%\begin{frame}{Before vs. After (by form)}
%  \begin{columns}[T,onlytextwidth]
%    \column{0.5\textwidth}
%      \begin{figure}
%        \includegraphics[width=\linewidth]{baseline_by_form.pdf}
%        \caption{Baseline by form.}
%      \end{figure}
%    \column{0.5\textwidth}
%      \begin{figure}
%        \includegraphics[width=\linewidth]{remove_expletives_by_form.pdf}
%        \caption{After removing expletives.}
%      \end{figure}
%  \end{columns}
%  \vspace{0.4em}
%  \begin{itemize}
%    \item Removing expletives reduces variance in sparse categories \parencite{roe2022power}.
%    \item Visual comparison follows best practices for paired panels \parencite{tufte1990}.
%  \end{itemize}
%\end{frame}

% ==================================================================
% CHAPTER 2: Transfer effects in bilingual acquisition of the null-subject constraint
% ==================================================================

% Chapter 2 overview
\begin{frame}
  \frametitle{Chapter 2: Transfer Effects in Bilingual Acquisition}
  \begin{beamercolorbox}[wd=\textwidth,sep=0.4em,colsep=0pt]{block body alerted}
    \textbf{Core Research Questions:}
    \begin{itemize}
    	\item<1-> Are large language models capable of maintaining competence when trained multilingually?
    	\item<2-> Does the order of language presentation impact learnability of subject drop?
    	\item<3-> Do L1 transfer effects vary based on evidence strength?
    \end{itemize}
  \end{beamercolorbox}
  
  \vspace{0.8em}
  
  \onslide<4->{
  \textbf{Approach:}
  \begin{itemize}
    \item<4-> Sequential bilingual training: English $\leftrightarrow$ Italian
    \item<5-> 2 x 2 design: L1 language x Training duration
    \item<6-> Evaluate null-subject competence in both languages
  \end{itemize}
  }
  
  \vspace{0.5em}
\end{frame}

% Chapter 2 Experiments Overview
\begin{frame}
  \frametitle{Chapter 2: Four Bilingual Training Experiments}
  
  \textbf{Training Protocol: 90M Dataset} \\
  \textit{L1 training (1 or 2 epochs) $\rightarrow$ L2 training (5 epochs opposite language)}
  
  \vspace{1em}
  
  \begin{columns}[T,onlytextwidth]
    \column{0.48\textwidth}
    \textbf{English First Models:}
    \begin{enumerate}
      \setcounter{enumi}{0}
      \item \textbf{Experiment 0:} English 1 epoch $\rightarrow$ Italian 5 epochs
      \item \textbf{Experiment 1:} English 2 epochs $\rightarrow$ Italian 5 epochs
    \end{enumerate}
    
    \column{0.48\textwidth}
    \textbf{Italian First Models:}
    \begin{enumerate}
      \setcounter{enumi}{2}
      \item \textbf{Experiment 2:} Italian 1 epoch $\rightarrow$ English 5 epochs
      \item \textbf{Experiment 3:} Italian 2 epochs $\rightarrow$ English 5 epochs
    \end{enumerate}
  \end{columns}
  
  \vspace{0.8em}
  
  \onslide<5->{
  \begin{beamercolorbox}[wd=\textwidth,sep=0.3em,colsep=0pt]{block body alerted}
    \textbf{Key Question:} Is there asymmetric transfer/impairment between languages?
  \end{beamercolorbox}
  }
\end{frame}

\begin{frame}{Italian Stimuli: Pro-Drop Language}
  \textbf{Core Contrasts (Italian pro-drop)}
  
  \textbf{Person/Number}: Anna ha finito il libro. $\emptyset$/\%Lei pensa che il finale sia perfetto. \\
  \textit{Anna has finished the book. $\emptyset$/She thinks that the ending is perfect.}
  
  \vspace{0.5em}
  \textbf{Control}: Maria ha convinto suo fratello $\emptyset$/*lui a partire presto. \\
  \textit{Maria convinced her brother $\emptyset$/*him to leave early.}
  
  \vspace{0.5em}
  \textbf{Expletives}: $\emptyset$/*Sembra che gli studenti abbiano superato l'esame. \\
  \textit{$\emptyset$/*It seems that the students have passed the exam.}
  
  \vspace{0.5em}
  \textbf{Conjunctions}: Giovanni si è svegliato tardi e $\emptyset$/lui ha perso il treno. \\
  \textit{Giovanni woke up late and $\emptyset$/he missed the train.}
  
  \vspace{1em}
  \textit{Italian allows null subjects where English requires overt realization}
\end{frame}

% Default Account and Predictions
\begin{frame}
  \frametitle{The 'Default' Account and Predictions}
  
  \begin{beamercolorbox}[wd=\textwidth,sep=0.4em,colsep=0pt]{block title alerted}
    \textbf{Theoretical Background}
  \end{beamercolorbox}
  \begin{beamercolorbox}[wd=\textwidth,sep=0.3em,colsep=0pt]{block body alerted}
    Children initially default to null subjects, then learn overt subject requirements
  \end{beamercolorbox}
  
  \vspace{0.8em}
  
  \textbf{Asymmetry Prediction:}
  \begin{itemize}
    \item<1-> \textbf{Italian$\rightarrow$English:} Pro-drop L1 should not differentially impact English L2 learning
    \item<2-> \textbf{English$\rightarrow$Italian:} Non-pro-drop L1 should impair Italian L2 learning
    \item<3-> \textbf{Result:} Italian-first models outperform English-first models
  \end{itemize}
  
  \vspace{0.8em}
  
  \onslide<4->{
  \begin{beamercolorbox}[wd=\textwidth,sep=0.3em,colsep=0pt]{block body alerted}
    \textbf{Critical Test:} Do we see greater transfer benefits vs. interference costs?
  \end{beamercolorbox}
  }
\end{frame}

% Chapter 2 Methodology
\begin{frame}
  \frametitle{Training Methodology: Four Bilingual Models}
  
  \begin{beamercolorbox}[wd=\textwidth,sep=0.4em,colsep=0pt]{block title alerted}
    \textbf{Training Protocol}
  \end{beamercolorbox}
  \begin{beamercolorbox}[wd=\textwidth,sep=0.3em,colsep=0pt]{block body alerted}
    Four models trained for 6-7 epochs total with systematic checkpointing
  \end{beamercolorbox}
  
  \vspace{0.8em}
  
  \textbf{Checkpoint Strategy:}
  \begin{itemize}
    \item<1-> \textbf{Log-step checkpoints} within first epoch: 1, 2, 4, 8, 16, 32...
    \item<2-> \textbf{Regular checkpoints} throughout L1 and L2 training
    \item<3-> \textbf{Final 5 epochs} (L2 phase) densely sampled for analysis
  \end{itemize}
  
  \vspace{0.8em}
  
  \onslide<4->{
  \textbf{Evaluation Across Training:}
  \begin{itemize}
    \item<4-> Models evaluated on both English and Italian stimuli
    \item<5-> Learning curves constructed for L1 and L2 acquisition
    \item<6-> Age of Acquisition (AoA) and end-state performance measured
  \end{itemize}
  }
\end{frame}

\begin{frame}
  \frametitle{Predictions: Asymmetric Transfer Effects}
  
  \textbf{Key Hypothesis:}
  \begin{itemize}
    \item<1-> \textbf{English background} should \emph{impair} Italian L2 learning
    \item<2-> \textbf{Italian background} should \emph{not impair} English L2 learning
    \item<3-> Asymmetry reflects different constraint directions
  \end{itemize}
  
  \vspace{0.8em}
  
  \onslide<4->{
  \textbf{Specific Predictions:}
  
  \begin{columns}[T,onlytextwidth]
    \column{0.48\textwidth}
      \textbf{English L1 Models:}
      \begin{itemize}
        \item More English training $\rightarrow$ slower Italian L2
        \item Delayed AoA for Italian null subjects
        \item Stronger interference effects
      \end{itemize}
      
    \column{0.48\textwidth}
      \textbf{Italian L1 Models:}
      \begin{itemize}
        \item Amount of Italian training has minimal effect
        \item Consistent English L2 acquisition
        \item No systematic interference
      \end{itemize}
  \end{columns}
  }
  
%  \vspace{0.8em}
%  
%  \onslide<7->{
%  \begin{beamercolorbox}[wd=\textwidth,sep=0.3em,colsep=0pt]{block body alerted}
%    \textbf{Test:} Compare L2 learning curves across 1 vs. 2 epoch L1 conditions
%  \end{beamercolorbox}
%  }
\end{frame}


% ==================================================================
% CHAPTER 3: The syntactic priming of null-subjects cross-linguistically
% ==================================================================

% Chapter 3 overview
\begin{frame}
  \frametitle{Chapter 3: Cross-Linguistic Priming of Null Subjects}
  \begin{beamercolorbox}[wd=\textwidth,sep=0.4em,colsep=0pt]{block body example}
    \textbf{Research Questions:}
    \begin{itemize}
    	\item<1-> Do large language models form cross-linguistic abstract representations?
    	\item<2-> Can structural priming reveal competence beyond performance?
    	\item<3-> How do bilingual models represent the 'absence' of subjects?
    \end{itemize}
  \end{beamercolorbox}
  
  \vspace{0.8em}
  
  \onslide<4->{
  \textbf{Method:}
  \begin{itemize}
    \item<4-> Prime with null/overt subjects in Language A
    \item<5-> Measure surprisal on target verbs in Language B
    \item<6-> Compare cross-linguistic priming effects
  \end{itemize}
  }
  
  \vspace{0.5em}
\end{frame}

% Chapter 3 background
\begin{frame}
  \frametitle{Cross-Linguistic Structural Priming}
  
  \begin{beamercolorbox}[wd=\textwidth,sep=0.4em,colsep=0pt]{block title example}
    \textbf{Theoretical Motivation}
  \end{beamercolorbox}
  \begin{beamercolorbox}[wd=\textwidth,sep=0.3em,colsep=0pt]{block body example}
    If models have abstract syntactic representations, they should show cross-linguistic priming
  \end{beamercolorbox}
  
  \vspace{0.8em}
  
  \textbf{Key Innovation:}
  \begin{itemize}
    \item<1-> Priming reveals representations that performance tasks cannot
    \item<2-> Cross-linguistic design eliminates lexical confounds
    \item<3-> Measures abstract 'absence' of subjects across languages
  \end{itemize}
  
  \vspace{0.8em}
  
  \onslide<4->{
  \textbf{Prediction:}
  \begin{itemize}
    \item<4-> Null subjects in Language A prime null preference in Language B
    \item<5-> Overt subjects in Language A prime overt preference in Language B
    \item<6-> Effects independent of lexical overlap or surface similarity
  \end{itemize}
  }
\end{frame}

% Priming Method Frame
\begin{frame}
  \frametitle{Structural Priming Methodology}
  
  \begin{beamercolorbox}[wd=\textwidth,sep=0.4em,colsep=0pt]{block title example}
    \textbf{Surprisal-Based Measurement}
  \end{beamercolorbox}
  \begin{beamercolorbox}[wd=\textwidth,sep=0.3em,colsep=0pt]{block body example}
    Following Sinclair et al. (2022) and Momma et al. (2025)
  \end{beamercolorbox}
  
  \vspace{0.8em}
  
  \textbf{Calculation:}
  \begin{itemize}
    \item<1-> Prime: Null/overt subject in Language A
    \item<2-> Target: Verb prediction in Language B
    \item<3-> Compare: -logProb(verb|null\_prime) - -logProb(verb|overt\_prime)
  \end{itemize}
  
  \vspace{0.8em}
  
  \onslide<4->{
  \textbf{Critical Insight:}
  \begin{itemize}
    \item<4-> No lexical overlap between prime and target
    \item<5-> No shared surface structure between languages
    \item<6-> Pure test of abstract syntactic representation
  \end{itemize}
  }
  
  \vspace{0.5em}
  
%  \onslide<7->{
%  \begin{beamercolorbox}[wd=\textwidth,sep=0.3em,colsep=0pt]{block body example}
%    \textbf{Result:} Competence measurable even when performance tasks fail
%  \end{beamercolorbox}
%  }
\end{frame}

% Cross-Linguistic Priming Design
\begin{frame}
  \frametitle{From Parallel Stimuli to Cross-Linguistic Priming}
  
  \begin{beamercolorbox}[wd=\textwidth,sep=0.4em,colsep=0pt]{block title example}
    \textbf{Leveraging Bilingual Evaluation Sets}
  \end{beamercolorbox}
  \begin{beamercolorbox}[wd=\textwidth,sep=0.3em,colsep=0pt]{block body example}
    Transform parallel English/Italian stimuli into priming experiments
  \end{beamercolorbox}
  
  \vspace{0.8em}
  
  \textbf{Standard Parallel Evaluation:}
  \begin{itemize}
    \item<1-> English: \textit{Anna finished. She/*$\emptyset$ thinks...}
    \item<2-> Italian: \textit{Anna ha finito. $\emptyset$/Lei pensa...}
    \item<3-> Separate within-language comparisons
  \end{itemize}
  
  \vspace{0.8em}
  
  \onslide<4->{
  \textbf{Cross-Linguistic Priming Design:}
  \begin{itemize}
    \item<4-> \textbf{Cross-join} English and Italian sentences
    \item<5-> Prime with Italian \textit{null} $\rightarrow$ Target English verb
    \item<6-> Prime with Italian \textit{overt} $\rightarrow$ Target English verb
    \item<7-> Compare surprisal differences across prime conditions
  \end{itemize}
  }
\end{frame}

\begin{frame}
  \frametitle{Cross-Linguistic Priming Matrix}
  
  \textbf{Experimental Design:}
  
  \vspace{0.5em}
  
  \begin{center}
  \begin{tabular}{l|c|c}
    \textbf{Prime Language} & \textbf{Target Language} & \textbf{Measurement} \\
    \hline
    Italian null & English & Surprisal on English verb \\
    Italian overt & English & Surprisal on English verb \\
    \hline
    English null & Italian & Surprisal on Italian verb \\
    English overt & Italian & Surprisal on Italian verb \\
  \end{tabular}
  \end{center}
  
  \vspace{0.8em}
  
  \textbf{Key Advantages:}
  \begin{itemize}
    \item<1-> Syntactic priming should occur with \textbf{no lexical overlap} between prime and target
    \item<2-> Simple stimuli construction, as we can use any parallel eval set to assess abstract representations as well as preferences
  \end{itemize}
\end{frame}

\begin{frame}
  \frametitle{Priming as a Window into Abstract Syntax}
  
  \begin{beamercolorbox}[wd=\textwidth,sep=0.4em,colsep=0pt]{block title example}
    \textbf{What Cross-Linguistic Priming Reveals}
  \end{beamercolorbox}
  \begin{beamercolorbox}[wd=\textwidth,sep=0.3em,colsep=0pt]{block body example}
    Abstract syntactic knowledge that transcends surface linguistic differences
  \end{beamercolorbox}
  
  \vspace{0.8em}
  
  \textbf{Theoretical Predictions:}
  \begin{itemize}
    \item<1-> If the model's are not showing robust cross-linguistic priming effects, that indicates that they are developing generalizations in a shallow, concrete way.
    \item<2-> If the model is forming strong syntactic generalizations, but these are not in the same direction as would be expected for human learners, that is problematic for an approach relying on a Platonic assumption
  \end{itemize}
  
  \vspace{0.8em}
  
  \onslide<4->{
  \textbf{Expected Results:}
  \begin{itemize}
    \item<4-> Italian null subjects prime English null preference
    \item<5-> Italian overt subjects prime English overt preference  
    \item<6-> and Vice Versa
  \end{itemize}
  }
  
  \vspace{0.5em}
  
  \onslide<8->{
  \begin{beamercolorbox}[wd=\textwidth,sep=0.3em,colsep=0pt]{block body example}
    \textbf{Innovation:} First test of cross-linguistic null-subject priming in LLMs
  \end{beamercolorbox}
  }
\end{frame}

% ------------------------------------------------------------------
% References slide (auto-breaks across slides if long)
% ------------------------------------------------------------------
\begin{frame}[allowframebreaks]{References}
  \printbibliography
\end{frame}

\end{document}